\documentclass[landscape, 8pt]{extarticle}
\usepackage{geometry}
% \usepackage{showframe}
\usepackage[dvipsnames]{xcolor}

\colorlet{colour1}{Red}
\colorlet{colour2}{Green}
\colorlet{colour3}{Cerulean}

\geometry{
	a4paper, 
	margin=0.17in
}

\pretolerance=0
\hyphenpenalty=0

\usepackage{lmodern}

\usepackage[fontsize=7pt]{scrextend}

\usepackage{graphicx} % Required for inserting images
\usepackage{amsmath}
\usepackage{amsfonts}
\usepackage{amssymb}
% \usepackage{preamble}
\usepackage{enumitem}
\usepackage{multicol}
\usepackage{lipsum}
\usepackage[framemethod=TikZ]{mdframed}
% \usepackage{../thmboxes_white}
\usepackage{../thmboxes_v2}
\usepackage{float}
% \usepackage{setspace}
\usepackage[nodisplayskipstretch]{setspace}





% \setlength{\parskip}{0pt}

% Custom Definitions of operators
% \DeclareMathOperator{\im}{im}
% \DeclareMathOperator{\Fix}{Fix}
% \DeclareMathOperator{\Orb}{Orb}
% \DeclareMathOperator{\Stab}{Stab}
% \DeclareMathOperator{\send}{send}
\DeclareMathOperator{\dom}{dom}
% \DeclareMathOperator{\Maps}{Maps}
% \DeclareMathOperator{\sgn}{sgn}
% \DeclareMathOperator{\Mat}{Mat}
% \DeclareMathOperator{\scale}{sc}
% \DeclareMathOperator{\Hom}{Hom}
% \DeclareMathOperator{\id}{id}
% \DeclareMathOperator{\rk}{rk}
% \DeclareMathOperator{\Tr}{tr}
% \DeclareMathOperator{\diag}{diag}
% \DeclareMathOperator{\can}{can}

\usepackage{hyperref} % note: this is the final package

\parindent = 0pt

\renewcommand\labelitemi{\tiny$\bullet$}

\begin{document}

\setlength{\abovedisplayskip}{3.5pt}
\setlength{\belowdisplayskip}{3.5pt}
\setlength{\abovedisplayshortskip}{3.5pt}
\setlength{\belowdisplayshortskip}{3.5pt}

\begin{multicols}{3}
\raggedcolumns


\section*{\huge Exam Notes}
Made by Leon :) \textit{Note: Any reference numbers are to the lecture notes}

\vspace{-5pt}
\section{Revisiting FPM}

\begin{dfn}[Nested Sequences]{dfn:nested-sequence}{1.1}
	A sequence $(I_{n})_{n\in\mathbb{N}}$ of sets is said to be \textbf{nested} if
	\[I_{1} \subset I_{2} \subset I_{3} \subset \cdots\]
\end{dfn}

\begin{thm}[Nested Interval Property]{thm:nested-interval-property}{1.1}
	If $(I_{n})$ is a nested sequence of nonempty closed bounded intervals then
	$$E = \bigcap\limits_{n\in\mathbb{N}} I_{n} = \{x\in\mathbb{R}: x\in I_{n},\,\forall n\in\mathbb{N}\}$$
	is nonempty (i.e. it contains at least one number). Moreover if $\lambda(I_{n})\to 0$, where $\lambda(I_{n})$ denotes the length of interval $I_{n}$, then $E$ contains exactly one number
\end{thm}

\begin{thm}[Covers]{thm:finite-subcovers}{1.2}
	Let $E$ be a subset of $\mathbb{R}^n$
	\begin{itemize}
		\setlength\itemsep{0em}
		\item A \textbf{cover} of $E$ is a collection of sets  $\{I_{\alpha}\}_{\alpha\in A}$ such that
			\[E\subseteq \bigcup\limits_{\alpha\in A} I_{\alpha}\]
		\item An \textbf{open covering} of $E$ is a cover such that each $I_{\alpha}$ is open, i.e.$(a,b)$ compared to $[a,b]$
		\item A \textbf{finite subcover} of $E$ is a collection of sets $(I_{\alpha})_{\alpha\in A_{0}}$ where there exists a subset $A_{0} = \{\alpha_{1}, \alpha_{2}, \dots, a_{N}\}$ of $A$ such that $(I_{\alpha})_{\alpha\in A_{0}}$ is a finite subset of $(I_{\alpha})_{\alpha\in A}$ that is also a cover
		\item The set $E$ is said to be \textbf{compact} iff every open covering of $E$ has a \textbf{finite subcovering}; that is
			\[E\subseteq \bigcup\limits_{j=1}^{N} I_{aj} \quad \text{ or } \quad E \subseteq I_{\alpha_{1}} \cup I_{a_{2}} \cup \cdots \cup I_{a_{N}}\]
	\end{itemize}
\end{thm}

\begin{dfn}[Epsilon-N Convergence of Sequence]{dfn:epsilon-n-sequences}{1.2}

	A sequence of real numbers $(x_{n})$ is said to \textbf{converge} to a real number $a\in \mathbb{R}$ iff for every $\epsilon>0$ there is an $N\in\mathbb{N}$ such that

	\[n\ge N \text{ implies } \lvert x_{n} - a \rvert < \epsilon\]
	If $(x_{n})$ converges to $a$, we will write $\displaystyle\lim_{n \to \infty} x_{n}=a$, or $x_{n}\to a$. The number $a$ is called the limit of the sequence $(x_{n})$. A sequence that does not converge to some real number is said to *diverge
\end{dfn}

\begin{dfn}[Cauchy Sequence]{dfn:cauchy-sequence}{1.3}
	A sequence $(x_{n})$ of numbers $x_{n} \in \mathbb{R}$ is said to be \textbf{Cauchy} if for every $\epsilon > 0$ there is $N\in \mathbb{N}$ such that
	\[\lvert x_{n} - x_{m} \rvert < \epsilon \quad \forall n,m\ge N\]
\end{dfn}

\begin{thm}[Convergent Sequences are Cauchy]{thm:convergent-seq-cauchy}{1.3}
	Let $(x_{n})$ be a sequence of real numbers. Then $(x_{n})$ is a Cauchy sequence if and only if $(x_{n})$ is a convergent sequence.

	\textbf{Note}: This works both ways ($(x_{n})$ is a convergent seq $\implies$ Cauchy)
\end{thm}

\begin{dfn}[Subsequences]{dfn:subsequence}{1.4}
	Suppose $(x_{n})_{n\in\mathbb{N}}$ is a sequence. A subsequence of this sequence is a sequence of the form $(x_{n_{k}})_{k\in\mathbb{N}}$ where for each $k$ there is a positive integer $n_{k}$ such that

	\[n_{1} < n_{2} < \cdots < n_{k} < n_{k+1} < \cdots\]
	Thus, $(x_{n})_{n\in\mathbb{N}}$ is just a selection of some (possibly all) of the $x_{n}$'s taken in order
\end{dfn}

\begin{thm}[Bolzano-Weierstrass]{thm:bolzano-weierstrass-thm}{1.5}
	Every bounded sequence of real numbers has a convergent subsequence
\end{thm}


\begin{dfn}[Limit Superior and Inferior]{dfn:limsup-liminf}{1.5}
	If $(x_{n})$ is a bounded sequence of real numbers we denote by

	\[\limsup_{{n\to\infty}} x_{n} = \lim_{n \to \infty} \left(\displaystyle \sup_{k\ge n} x_{k}\right),\,\qquad \liminf_{{n\to\infty}} x_{n} = \lim_{n \to \infty} \left(\displaystyle \inf_{k\ge n} x_{k}\right)\]

	\noindent\rule{\textwidth}{0.2pt}
	\textbf{Note}: These are only defined for bounded sequences

	\begin{itemize}
		\setlength\itemsep{0em}
		\item If $(x_{n})$ is not bounded from above then we write $\limsup_{n \to \infty} x_{n} = +\infty$

		\item If $(x_{n})$ is not bounded from below then we write $\liminf_{n \to \infty} x_{n} = +\infty$
	\end{itemize}
\end{dfn}

\begin{thm}[Convergence from Limsup and Liminf]{thm:limsup-convergence}{1.6}
	A sequence $(x_{n})$ of real numbers is convergent if and only if $\limsup_{n \to \infty}x_{n}$ and $\liminf_{n \to \infty}x_{n}$ are real numbers and

	\[\limsup_{n \to \infty} x_{n} = \liminf_{n \to \infty} x_{n}\]
\end{thm}

\begin{dfn}[Convergent Infinite Series]{dfn:convergent-infinite-series}{1.6}
	Let $S=\sum_{k=1}^{\infty}a_{k}$ be an infinite series $a_{k}$. For each $n\in\mathbb{N}$, the partial sum of $S$ of order $n$ is defined by

	\[s_{n} = \sum_{k=1}^{n} a_{k}\]
	$S$ is said to \textbf{converge} iff its sequence of partial sums $(s_{n})$ converges to some $s \in\mathbb{R}$ as $n\to\infty$; that is, iff for every $\epsilon>0$ there is an $N\in\mathbb{N}$ such that for all $n\ge N$ we have $\lvert s_{n}-s \rvert < \epsilon$. In this case we shall write

	\[\sum_{k=1}^{\infty} a_{k} = s\]
	and call $s$ the \textbf{sum} or \textbf{value} of the series $\sum_{k=1}^{\infty}a_{k}$

	\noindent\rule{\textwidth}{0.2pt}

	A series $S=\sum_{k=1}^{\infty}a_{k}$ is said to be \textbf{absolutely convergent} if the series $\sum_{k=1}^{\infty}\lvert a_{k} \rvert$ is convergent. A series is called \textbf{conditionally convergent} if it is convergent but not absolutely convergent.
\end{dfn}


% idk where i got this guy actually?

% \begin{thm}[Cauchy Criteron]{thm:cauchy-criteron}{1.7}
% 	Let $\{a_{k}\}$ be a real sequence. Then the infinite series $\sum_{k=1}^{\infty} a_{k}$ converges if and only if given $\epsilon>0$ there is an $N\in\mathbb{N}$ such that
%
% 	\[n\ge N \text{ implies } \left\lvert  \sum_{k=n}^{\infty} a_{k}  \right\rvert <\epsilon\]
% \end{thm}

\begin{thm}[Cauchy Criteron]{thm:cauchy-criteron}{1.7}
	Let $S=\sum_{k=1}^{\infty}a_{k}$ be a series. Then the series $S$ is convergent iff for any $\epsilon>0$ there exists $N$ such that for all $m\ge n\ge N$ we have that

	\[\left\lvert  \sum_{k=n+1}^{m} a_{k}  \right\rvert < \epsilon\]
\end{thm}


\begin{thm}[Rearrangements of Abs. Convergent Series]{thm:rearrangements-absolute-convergent-series}{1.8}
	Let $S=\sum_{k=1}^{\infty} a_{k}$ be an absolutely convergent series. Then

	\begin{itemize}
		\setlength\itemsep{0em}
		\item The series $S$ is convergent

		\item Let $z:\mathbb{N}\to \mathbb{N}$ be a bijection. Then the series $\sum_{k=1}^{\infty} a_{z(k)}$ is convergent and
			\[\sum_{k=1}^{\infty} a_{k} = \sum_{k=1}^{\infty} a_{z(k)} \]
	\end{itemize}

	The series $\sum_{k=1}^{\infty} a_{z(k)}$ is called a \textbf{rearrangement} of the series $\sum_{k=1}^{\infty} a_{k}$. What we do here is add the terms of the sum in a different order to the original one, for example

	\[a_{3} + a_{7}+ a_{1}+ a_{100} + a_{2} + \dots\]
	Since $z:\mathbb{N}\to \mathbb{N}$ is a bijection, we will miss no terms.

\end{thm}

\newpage
\begin{thm}[Rearrangements of Cond. Convergent Series]{thm:rearrangements-conditionally-convergent-series}{1.9}
	Let $S = \sum_{k=1}^{\infty} a_{k}$ be any conditionally convergent series. Then there exists rearrangements $z:\mathbb{N}\to \mathbb{N}$ (where $z$ is a bijection) such that

	\begin{itemize}
		\setlength\itemsep{0em}
		\item For any $r\in\mathbb{R}$ the series $\sum_{k=1}^{\infty} a_{z(k)}$ is conditionally convergent and its sum is $r$

		\item The series $\sum_{k=1}^{\infty} a_{z(k)}$ diverges to $+\infty$

		\item The series $\sum_{k=1}^{\infty} a_{z(k)}$ diverges to $-\infty$

		\item The partial sums of the series $\sum_{k=1}^{\infty} a_{z(k)}$ oscillate between any two real numbers

	\end{itemize}
\end{thm}

% TODO: section 1.4

\begin{dfn}[Continuity]{dfn:continuity}{1.7}
	Let $f$ be a function $f : \dom(f) \to \mathbb{R}$ where $\dom(f)\subset \mathbb{R}$. We say that $f$ is continuous at some $a\in \dom(f)$ if for any sequence $(x_{n})$ whose terms lie in $\dom(f)$ and which converges to $a$, we have $\lim_{n\to \infty} f(x_{n}) = f(a)$. If $f$ is continuous at each $a\in S \subset \dom(f)$ then we say $f$ is continuous on $S$. If $f$ is continuous of $\dom(f)$ then we say $f$ is continuous
\end{dfn}

\begin{thm}[Properties of Continuity]{thm:continuity-props}{1.10}
	Let $f, g : D \to \mathbb{R}$ be continuous on $D$, and let $\alpha\in \mathbb{R}$. Then the following functions are continuous on $D:$
	\vspace{-10pt}
	\begin{multicols}{3}
		\begin{enumerate}
			\item $\alpha$ f
			\item $f + g$
			\item $fg$
		\end{enumerate}
	\end{multicols}
\end{thm}

\begin{dfn}[Composition]{dfn:composition}{1.8}
	Let $A, B \subseteq \mathbb{R}$ be nonempty, let $f : A \to \mathbb{R},\, g : B \to \mathbb{R}$ and $f(A) \subseteq B$. The composition of $g$ with $f$ is the function $g \circ f : A \to \mathbb{R}$ defined by
	\[(g \circ f)(x) = g(f(x)), \quad \text{for all $x\in A$}\]
\end{dfn}

\begin{thm}[Continuity of Composition]{thm:composition-continuity}{1.11}
	If $f$ is continuous at $a\in \mathbb{R}$ and $g$ is continuous at $f(a)$ then the composition $g \circ f$ is continuous at $a$
\end{thm}

\begin{thm}[\texorpdfstring{$\epsilon-\delta$}{epsilon-delta} definition of continuity]{thm:e-d-def-continuity}{1.12}
	Let $f$ be a function $f : \dom(f) \to \mathbb{R}$ where $\dom(f) \subset \mathbb{R}$. Then $f$ is continuous at $a\in \dom(f)$ iff for any $\epsilon > 0$ there exists $\delta > 0$ s.t. whenever $x\in\dom(f)$ and $\lvert x - a \rvert < \delta$ we have $\lvert f(x) - f(a) \rvert < \epsilon$
\end{thm}

\begin{dfn}[Intermediate Value Theorem]{dfn:intermediate-value-thm}{1.13}
	Let $a < b$ real numbers and $f : [a, b] \to \mathbb{R}$ be continuous on $[a, b]$. If $f(a)f(b)<0$ then there exists at least one $c\in (a, b)$ s.t. $f(c) = 0$
\end{dfn}

\begin{dfn}[Extreme Value Theorem]{dfn:extreme-value-thm}{1.14}
	Let $a < b$ real numbers and $f : [a, b]\to \mathbb{R}$ be continuous on $[a, b]$. Then there exists points $c, d\in [a,b]$ s.t.
	\[f(c) = \inf \{f(x) : x\in [a, b]\}, \quad f(d) = \sup \{f(x) : x\in [a, b]\}\]
	That is, the function $f$ on the interval $[a, b]$ is bounded and attains its minimal value at some point $c\in [a,b]$. Similarly, the maximal value of $f$ is also attained at some point $d\in [a,b]$
\end{dfn}


\section{Uniform convergence}

\begin{dfn}[Pointwise Convergence]{dfn:pointwise-convergence}{2.1}
	Let $E$ be a nonempty subset of $\mathbb{R}$. A sequence of functions $f_{n}: E\to \mathbb{R}$ is said to \textbf{converge pointwise} on $E$, written $f_{n}\to f$ pointwise on $E$ as $n\to \infty$, iff $f(x) = \displaystyle\lim_{n \to \infty}f_{n(x)}$ exists for each $x \in E$

	\noindent\rule{\textwidth}{0.2pt}
	Let $E$ be a nonempty subset of $\mathbb{R}$. Then a sequence of functions $f_{n}$ converges pointwise on $E$, as $n\to\infty$, iff for every $\epsilon>0$ and $x \in E$ there is an $N \in\mathbb{N}$ (which may depend on $x$ as well we $\epsilon$) such that

	\[n\ge N\quad\text{implies}\quad \lvert f_{n}(x)-f(x) \rvert < \epsilon\]

	\noindent\rule{\textwidth}{0.2pt}
	\textbf{Remarks}:
	\begin{itemize}
		\setlength\itemsep{0em}
		\item The pointwise limit of continuous (respectively, differentiable) functions is not necessarily continuous (respectively, differentiable).

		\item The pointwise limit of integrable functions is not necessarily integrable.

		\item There exist continuous functions $f_{n}$ and $f$ such that $f_{n}\to f$ pointwise on $[0,1]$ but

			\[\lim_{n \to \infty} \int_{0}^{1} f_{n}(x) \, dx \ne \int_{0}^{1} \left(\lim_{n \to \infty} f_{n}(x)\right) \, dx \]
	\end{itemize}

\end{dfn}


\begin{dfn}[Uniform Convergence]{dfn:uniform-convergence}{2.2}
	Let $E$ be a nonempty subset of $\mathbb{R}$. A sequence of functions $f_{n}: E\to\mathbb{R}$ is said to \textbf{converge uniformly} on $E$ to a function $f$ (notation: $f_{n}\to f$ uniformly on $E$ as $n\to\infty$) if and only if for every $\epsilon>0$ there is an $N \in\mathbb{N}$ such that for all $x \in E$

	\[n\ge N\quad\text{implies}\quad \lvert f_{n}(x)-f(x) \rvert <\epsilon\]
\end{dfn}


\begin{rem}[Differences between Pointwise and Uniform]{rem:pointwise-uniform-diffs}{2.2}
	Let $E$ be a nonempty subset of $\mathbb{R}$. 

	\begin{itemize}
		\setlength\itemsep{0em}
		\item A sequence of functions $f_{n}$ \textbf{converges pointwise} on $E$, as $n\to\infty$, if and only if for every $\epsilon>0$ and $\textcolor{red}{x \in E}$ there is an $N \in\mathbb{N}$ (which may depend on $x$ as well we $\epsilon$) such that

			\[n\ge N\quad\text{implies}\quad \lvert f_{n}(x)-f(x) \rvert < \epsilon\]

		\item A sequence of functions $f_{n}: E\to\mathbb{R}$ \textbf{converges uniformly} on $E$ iff for every $\epsilon>0$ there is an $N \in\mathbb{N}$ such that for all $\textcolor{red}{x \in E}$

			\[n\ge N\quad\text{implies}\quad \lvert f_{n}(x)-f(x) \rvert <\epsilon\]
	\end{itemize}

	For a sequence of functions to be pointwise convergent, it is enough to have an $N_{n}$ for every $x_{n}$, but for it to be uniformly convergent, it has to have \textbf{the same} $N$ for every $x$ in the sequence

\end{rem}


\begin{thm}[Equivalence of Uniform Convergence]{thm:uniform-convergence-equivalence}{2.1}
	The following are equivalent concerning a sequence of functions $f_{n}:E\to \mathbb{R}$ and $f: E\to \mathbb{R}$:

	\begin{itemize}[leftmargin=*]
		\setlength\itemsep{0em}
		\item $f_{n}\to f$ uniformly on $E$

		\item $\displaystyle\sup_{x\in E}\lvert f_{n}(x)-f(x) \rvert\to 0$ as $n\to\infty$

		\item there exists a sequence $a_{n}\to 0$ s.t. $\lvert f_{n}(x)-f(x) \rvert\le a_{n},\, \forall x\in E$

	\end{itemize}
\end{thm}


\begin{thm}[]{thm:continuity-of-convergent-f}{2.1}
	Let $E$ be a nonempty subset of $\mathbb{R}$ and suppose that $f_{n}\to f$ uniformly on $E$ as $n\to\infty$. If each $f_{n}$ is continuous at some $x_{0}\in E$, then $f$ is continuous at $x_{0}\to E$
\end{thm}

\begin{dfn}[Uniformly Bounded Sequences]{dfn:uniformly-bounded}{2.2}
	A sequence of functions $f_{n}$ is said to be \textbf{uniformly bounded} on a set $E$ if there is a $M>0$ such that $\lvert f_{n}(x) \rvert\le M$ for all $x\in E$ and all $n\in N$

\end{dfn}

\begin{thm}[]{thm:uniform-convergence-limit-swap}{2.2}
	Suppose that $f_{n}\to f$ uniformly on a closed interval $[a,b]$. If each $f_{n}$ is integrable on $[a,b]$, then so is $f$ and 

	\[\lim_{n \to \infty} \int_{a}^{b} f_{n}(x) \, dx =\int_{a}^{b} \left(\lim_{n \to \infty} f_{n}(x)\right) \, dx \]
\end{thm}

\begin{thm}[]{thm:uniformly-convergent-derivatives}{2.3}
	Let $(a,b)$ be a bounded interval and suppose that $f_{n}$ is a sequence of functions which converges at some $x_{0}\in(a,b)$. If each $f_{n}$ is differentiable on $(a,b)$, and $f_{n}'$ converges uniformly on $(a,b)$ as $n\to\infty$, then $f_{n}$ converges uniformly on $(a,b)$ and

	\[\lim_{n \to \infty} f'_{n}(x)=\left(\lim_{n \to \infty} f_{n}(x)\right)'\]
\end{thm}

\newpage

\begin{dfn}[Convergence of series]{dfn:convergence-of-sequence}{2.3}
	Let $f_{k}$ be a sequence of a real functions defined on some set $E$ and set

	\[s_{n}(x)=\sum_{k=1}^{n} f_{k}(x),\quad x\in E,\,n\in \mathbb{N}\]
	\begin{itemize}
		\setlength\itemsep{0em}
		\item The series $\displaystyle\sum_{k=1}^{\infty} f_{k}$ is said to \textbf{converge pointwise} on $E$ if and only if the sequence $s_{n}(x)$ converges pointwise on $E$ as $n\to\infty$

		\item The series $\displaystyle\sum_{k=1}^{\infty} f_{k}$ is said to \textbf{converge uniformly} on $E$ if and only if the sequence $s_{n}(x)$ converges uniformly on $E$ as $n\to\infty$

		\item The series $\displaystyle\sum_{k=1}^{\infty} f_{k}$ is said to \textbf{converge absolutely} (pointwise) on $E$ if and only if $\displaystyle\sum_{k=1}^{\infty} \lvert f_{k}(x) \rvert$ converges for each $x\in E$

	\end{itemize}
\end{dfn}

\begin{thm}[Results of Convergent Series]{thm:convergent-series-results}{2.4}
	Let $E$ be a nonempty subset of $\mathbb{R}$ and let $(f_{k})$ be a sequence of real functions defined on $E$.

	\begin{itemize}
		\setlength\itemsep{0em}
		\item Suppose that $x_{0}\in E$ and that each $f_{k}$ is continuous at $x_{0}\in E$. If $f=\displaystyle\sum_{k=1}^{\infty}f_{k}$ converges uniformly on $E$, then $f$ is continuous at $x_{0}\in E$.

		\item Term-by-term integration: Suppose that $E=[a,b]$ and that each $f_{k}$ is integrable on $[a,b]$. If $f=\displaystyle\sum_{k=1}^{\infty} f_{k}$ converges uniformly on $[a,b]$, then $f$ is integrable on $[a,b]$ and

		\[\int_{a}^{b} \sum_{k=1}^{\infty} f_{k}(x) \, dx =\sum_{k=1}^{\infty} \int_{a}^{b} f_{k}(x) \, dx \]
		\item Term-by-term differentiation: Suppose that $E$ is a bounded, open interval and that each $f_{k}$ is differentiable on $E$. If $\sum_{k=1}^{\infty}f_{k}(x_{0})$ converges at some $x_{0}\in E$, and $g=\sum_{k=1}^{\infty}f'(k)$ converges uniformly on $E$, then $f= \sum_{k=1}^{\infty}f_{k}$ converges uniformly on $E$, is differentiable on $E$, and

		\[f'(x)=\left( \sum_{k=1}^{\infty} f_{k}(x) \right)'=\sum_{k=1}^{\infty} f'_{k}(x)=g(x)\]
		for $x\in E$

	\end{itemize}
\end{thm}

\begin{thm}[Weierstrass M-test]{thm:weierstrass-m-test}{2.5}
	Let $E$ be a nonempty subset of $\mathbb{R}$, let $f_{k}: E\to \mathbb{R},\,k\in\mathbb{N}$, and suppose that $M_{k}>0$ satisfies $\displaystyle\sum_{k=1}^{\infty} M_{k}<\infty$. If $\lvert f_{k}(x) \rvert \le M_{k}$ for $k\in \mathbb{N}$ and $x\in E$,  then $f=\displaystyle\sum_{k=1}^{\infty}f_{k}$ converges absolutely and uniformly on $E$.
\end{thm}

\begin{dfn}[Power Series]{dfn:power-series}{3.0}
	Let $(a_{n})$ be a sequence of real numbers, and $c\in \mathbb{R}$. A \textbf{power series} is a series of the form

	\[\sum_{n=0}^{\infty} a_{n}(x-c)^{n}\]
	The numbers $a_{n}$ are called the \textbf{coefficients} of the power series, and $c$ is its \textbf{centre}. In many cases it suffices to set $c=0$. Note that the series will always converge at the point $x=c$ as all terms beyond the first are $0$.
\end{dfn}


\begin{dfn}[Radius of Convergence]{dfn:radius-of-convergence}{3.1}
	The \textbf{radius of convergence} $R$ of the power series

	\begin{equation}
		\sum_{n=0}^{\infty} a_{n}(x-c)^{n}\tag{$*$}\label{$*$}
	\end{equation}
	is defined by

	\[R=\sup\{r\ge 0:(a_{n}r^n) \text{ is bounded}\}\]
	unless $(a_{n}r^{n})$ is bounded for all $r\ge 0$, in which case we say that $R=\infty$

	\noindent\rule{\textwidth}{0.2pt}

	\textbf{Thm 3.1}: Suppose the radius of convergence $R$ of \ref{$*$} satisfies $0<R<\infty$. If $\lvert x-c \rvert<R$, the power series \ref{$*$} converges absolutely. If $\lvert x-c \rvert>R$, the power series \ref{$*$} diverges
\end{dfn}

\begin{thm}[Continuty of Power Series]{thm:power-series-continuity}{3.2}
	Assume that $R>0$. Suppose that $0<r<R$. Then a power series converges uniformly and absolutely on $\lvert x-c \rvert\le r$ to a continuous function $f$. Hence

	\[f(x)=\sum_{n=0}^{\infty} a_{n}(x-c)^{n}\]
	defines a continuous function $f:(c-R,c+R)\to \mathbb{R}$

	\noindent\rule{\textwidth}{0.2pt}
	\textbf{Lemma 3.1}: The two power series

	\[\sum_{n=1}^{\infty} a_{n}(x-c)^{n}\text{ and } \sum_{n=1}^{\infty} na_{n}(x-c)^{n-1}\]
	have the same radius of convergence
\end{thm}

\begin{thm}[Differentiation of Power Series]{thm:power-series-differentiation}{3.3}
	Suppose the radius of convergence of a power series is $R$. Then the function

	\[f(x)=\sum_{n=0}^{\infty} a_{n}(x-c)^{n}\]
	is infinitely differentiable on $\lvert x-c \rvert<R$, and for such $x$,

	\[f'(x)=\sum_{n=0}^{\infty} na_{n}(x-c)^{n-1}\]
	and the series converges absolutely, and also uniformly on $[c-r,c+r]$ for any $r<R$. Moreover,

	\[a_{n}=\frac{f^{(n)}(c)}{n!}\]
\end{thm}

% TODO: maybe some stuff on exponential function (pg 32)


\newpage
\section{Lebesgue Integration}

\begin{dfn}[Characteristic Function]{dfn:characteristic-function}{4.0}
	Let $E$ be a subset of $\mathbb{R}$. We define its \textbf{characteristic function} $\chi_{E}:\mathbb{R}\to\mathbb{R}$ by $\chi_{E}(x)=1$ if $x\in E$ and $\chi_{E}(x)=0$ if $x\not\in E$. In other words,
	\[\chi_{E}=\begin{cases}
	1&x\in E \\
	0 & x\not\in E
	\end{cases}\]
	In other words, this is a function that is $1$ at all points of a bounded interval, and $0$ elsewhere

	\noindent\rule{\textwidth}{0.2pt}
	
	Let $I$ be a bounded interval with endpoints $a,\,b$ and $a\le b$. We call the number $b-a$ the \textbf{length of the interval} $I$ and we denote it by $\lambda(I)$. This might also be referred to as $\lvert I \rvert$. That is,
	\[\lambda((a,b)) = \lambda([a,b]) = \lambda((a,b]) = \lambda([a,b)) = b-a\]

	\noindent\rule{\textwidth}{0.2pt}
	From our definition of a characteristic function and the length of an interval, we have that the area of the characteristic function is a rectangle with width $\lambda(I)$ and height $1$, therefore
	\[\int \chi_{I} = 1 \cdot \lambda(I) = \lambda(I)\]
\end{dfn}

\begin{dfn}[Step function]{dfn:step-function}{4.1}
	We say that $\phi:\mathbb{R}\to \mathbb{R}$ is a \textbf{step function} if there exist real numbers $x_{0}<x_{1}<x_{2}<\cdots<x_{n}$ (for some $n\in \mathbb{N}$) such that

	\begin{enumerate}
	    \setlength\itemsep{0em}
	    \item $\phi(x)=0$ for $x<x_{0}$ and $x>x_{n}$
	    \item $\phi$ is constant on $(x_{j-1}, x_{j})$ for $1\le j \le n$
	\end{enumerate}

	We shall use the phrase "$\phi$ is a step function with respect to $\{x_{0},x_{1},\dots,x_{n}\}$" to describe this situation


	\noindent\rule{\textwidth}{0.2pt}

	\textbf{Properties of Step Functions}

	\begin{enumerate}
	    \setlength\itemsep{0em}
	    \item The class of step functions is a vector space - i.e. if $\phi$ and $\psi$ are step functions and $\alpha$ and $\beta$ are real numbers, then $\alpha\phi+\beta\psi$ is a step function, and that if $\phi$ and $\psi$ are step functions, then $\max\{\phi,\psi\}$, $\min\{\phi\psi\}$, $\lvert \phi \rvert$ and $\phi\psi$ are also step functions
	    \item If $\phi$ and $\psi$ are step functions, then $\phi + \psi$ is a step function
	    \item $\phi$ is a step function if and only if it is of the form
	\[\phi=\sum_{j=1}^{n} c_{j}\chi_{J_{j}}\]
	for some $n$, $c_{j}$, and bounded intervals $J_{j}$
	\end{enumerate}

	\noindent\rule{\textwidth}{0.2pt}

	\textbf{Def 4.2: Integral of a Step Function}

	If $\phi$ is a step function with respect to $\{x_{0},x_{1},\dots,x_{n}\}$ which takes the value $c_{j}$ on $(x_{j-1},x_{j})$, then
	\[\int \phi := \sum_{j=1}^{n} c_{j}(x_{j}-x_{j-1})\]
	Therefore, using the characteristic definition of a step function, the integral is
	\[\int \phi = \int \sum_{j=1}^{n} c_{j}\chi_{J_{j}}=\sum_{j=1}^{n} c_{j}\int \chi_{IJ_{j}}=\sum_{j=1}^{n} c_{j}\lambda(J_{j})\]
\end{dfn}

\begin{dfn}[Lebesgue Integrals]{dfn:lebesgue-integral}{4.3}
	A function $f:I\to \mathbb{R}$ is said to be \textbf{integrable} or more precisely \textbf{Lebesgue integrable} on an interval $I$ if there exist numbers $c_{j}$ and bounded intervals $J_{j}\subset I,\,j=1,2,3,\dots$ such that
	\[\sum_{j=1}^{\infty} \lvert c_{j} \rvert \lambda(J_{j})<\infty\]
	and the equality
	\[f(x)=\sum_{j=1}^{\infty} c_{j}\chi_{J_{j}}(x)\]
	holds for all $x\in I$ at which
	\[\sum_{j=1}^{\infty} \lvert c_{j} \rvert \chi_{J_{j}}(x)<\infty\]
	We denote by $\int_{I}f$ the number
	\[\int _{I}f=\sum_{j=1}^{\infty} c_{j}\lambda(J_{j})\]
	and call it the integral of $f$ over the interval $I$.
	If the function $f$ is not integrable on the interval $I$ then we say that the integral of $f$ on $I$ does not exist. Hence if we say that the integral of $f$ on $I$ exists it just means that $f$ is (Lebesgue) integrable on $I$.
\end{dfn}

\begin{thm}[Lebesgue Equality]{thm:lebesgue-equality}{4.1}
	Suppose that $c_{j}$, $d_{j}$ are real numbers and $J_{j}$, $K_{j}$ are bounded intervals for all $j=1,2,3,\dots$, and
	\[\sum_{j=1}^{\infty} \lvert c_{j} \rvert \lambda(J_{j})<\infty,\quad \sum_{j=1}^{\infty} \lvert d_{j} \rvert \lambda(K_{j})<\infty\]
	If
	\[\sum_{j=1}^{\infty} c_{j}\chi_{J_{j}}(x)=\sum_{j=1}^{\infty} d_{j}\chi_{K_{j}}(x)\]
	holds for all $x$ such that
	\[\sum_{j=1}^{\infty} \lvert c_{j} \rvert \chi_{J_{j}}(x)<\infty,\quad \sum_{j=1}^{\infty} \lvert d_{j} \rvert \chi_{K_{j}}(x)<\infty\]
	Then
	\[\sum_{j=1}^{\infty} c_{j}\lambda(J_{j})=\sum_{j=1}^{\infty} d_{j}\lambda(K_{j})\]
\end{thm}

\newpage
\begin{thm}[Lebesgue Integral Properties]{dfn:lebesgue-props}{4.2}
	Suppose $f$ and $g$ are integrable on $I$ and $\alpha$ and $\beta$ are real numbers. Then
	\begin{enumerate}
		\setlength\itemsep{0em}
		\item $\alpha f + \beta g$ is integrable on $I$ and
		\[\int_{I} (\alpha f + \beta g)=\alpha\int_{I} f+\beta\int_{I} g\]
		\item If $f\ge 0$ on $I$ then $\int_{I}f\ge 0$; if $f\ge g$ on $I$ then $\int_{I}f\ge \int_{I}g$
		\item $\lvert f \rvert$ is integrable on $I$ and $\left\lvert  \int_{f}f  \right\rvert\le \int_{I}\lvert f \rvert$
		\item $\max\{f,g\}$ and $\min\{f,g\}$ are integrable on $I$
		\item If one of the functions is bounded then the product $fg$ is integrable on $I$
		\item If $f\ge 0$ with $\int_{I}f=0$ then any function $h$ such that $0\le h\le f$ on $I$ is integrable on $I$
	\end{enumerate}
\end{thm}

\begin{thm}[Integrability of Sequences and Series]{thm:seq-series-integrability}{4.3}
	Suppose that $(f_{n})_{n\in\mathbb{N}}$ is a sequence of functions each of which is integrable on $I$
	\begin{enumerate}
		\setlength\itemsep{0em}
		\item Assume that
		\[\sum_{n=1}^{\infty} \int_{I} \lvert f_{n} \rvert <\infty\]
		Let $f$ be a function on the interval $I$ such that
		\[f(x)=\sum_{n=1}^{\infty} f_{n}(x)\quad\text{for all } x\in I\text{ such that}\quad \sum_{n=1}^{\infty} \lvert f_{n}(x) \rvert <\infty\]
		Then $f$ is integrable on $I$ and its integral on $I$ is equal to
		\[\int_{I} f=\sum_{n=1}^{\infty} \int_{I} f_{n}\]
		\item Assume that each $f_{n}\ge 0$ on $I$ and let $f(x)=\sum_{n=1}^{\infty}f_{n}(x)$ for all $x\in I$ (we allow for the possibility that at some points this sum is infinite). Then $f$ is integrable on $I$ if and only if
		\[\sum_{n=1}^{\infty} \int_{I} f_{n}<\infty\]
	\end{enumerate}
\end{thm}

\begin{thm}[Monotone Convergence Theorem]{thm:monotone-convergence}{4.4}
	Suppose that $(f_{n})$ is a monotone increasing sequence of integrable functions on an interval $I$. That is, $f_{1}(x) \le f_{2}(x) \le f_{3}(x) \le \dots$ for all $x\in I$. For all $x\in I$, let
	\[f(x) = \lim_{n\to \infty} f_{n}(x)\]
	where we allow for the possibility that at some points this limit is infinite. Then $f$ is integrable on $I$ iff
	\[sup_{n\in \mathbb{N}} \int_{I} f_{n} = \lim_{n\to \infty} \int_{I} f_{n} < \infty. \quad \text{Also,} \int_{I} f = \lim_{n\to \infty} \int_{I} f_{n}\]
\end{thm}



\begin{dfn}[Riemann Integrable Functions]{dfn:riemann-integrable}{4.4}
	Let $f : \mathbb{R} \to \mathbb{R}$. We say that $f$ is \textbf{Riemann-integrable} if for every $\epsilon > 0$ there exists step functions $\phi$ and $\psi$ such that
	\[\phi \le f \le \psi\]
	and
	\[\int \psi - \int \phi < \epsilon\]

	\noindent\rule{\textwidth}{0.2pt}
	\textbf{Thm 4.5}: A function $f : \mathbb{R}\to \mathbb{R}$ is Riemann-integrable iff
	\begin{align*}
		& \sup \left\{\int \phi : \text{$\phi$ is a step function and $\phi \le f$}\right\} \\
		= & \inf \left\{\int \psi : \text{$\psi$ is a step function and $\phi \ge f$}\right\}
	\end{align*}

	\noindent\rule{\textwidth}{0.2pt}
	\textbf{Def 4.5}: If $f$ is Riemann-integrable we define its Riemann integral $(R) \int f$ as the common value
	\begin{align*}
		(R) \int f&:= \sup \left\{\int \phi : \text{$\phi$ is a step function and $\phi \le f$}\right\} \\
		&= \inf \left\{\int \psi : \text{$\psi$ is a step function and $\phi \ge f$}\right\}
	\end{align*}
\end{dfn}

\begin{thm}[Connection between Riemann and Lebesgue]{thm:riemann-lebesgue-connection}{4.6}
	Suppose that $f : \mathbb{R} \to \mathbb{R}$ is Riemann-integrable. Then $f$ is also Lebesgue integrable on $\mathbb{R}$ and moreoever
	\[(R) \int f = \int f\]
	where the number on the lefthand side is the value of the Riemann integral of $f$, while the righthand side denotes the value of the Lebesgue integral of $f$ on $\mathbb{R}$
\end{thm}

\begin{thm}[Riemann lemmas]{thm:riemann-lemmas}{4.1}
	Let $f : \mathbb{R} \to \mathbb{R}$ be a bounded function with bounded support $[a, b]$. The following are equivalent:
	\begin{enumerate}
	    \item $f$ is Riemann-integrable
	    \item for every $\epsilon > 0$ there exists $a = x_{0} < \cdots < x_{n} = b$ s.t. if $M_{j}$ and $m_{j}$ denote the supremum and infimum values of $f$ on $(x_{j-1}, x_{j})$ respectively, then
			\[\sum_{j = 1}^{n}(M_{j} - m_{j})(x_{j} - x_{j - 1}) < \epsilon\]
		\item for every $\epsilon > 0$ there exists $\alpha = x_{0} < \cdots < x_{n} = b$ s.t. with $I_{j} = (x_{j-1}, x_{j})$ for $j \ge 1$
			\[\sum_{j = 1}^{n} \sup_{x,y\in I_{j}} \lvert f(x) - f(y) \rvert \lambda(I_{j}) < \epsilon\]
	\end{enumerate}

	\noindent\rule{\textwidth}{0.2pt}
	Notation to aid these lemmas: For $f : \mathbb{R}\to \mathbb{R}$ a bounded function with bounded support $[a, b]$ and for $a = x_{0} < \cdots < x_{n} = b$, we let $I_{j} = (x_{j - 1}, x_{j}),\, m_{j} := \inf_{x\in I_{j}}f(x)$ and $M_{j} := \sup_{x\in I_{j}} f(x)$. We define the \textbf{lower step function of $f$ with respect to $\{x_{0},\dots,x_{n}\}$} as
	\[\phi_{*}(x) = \sum_{j = 1}^{n} m_{j}\chi_{I_{j}}(x) + \sum_{j = 0}^{n} f(x_{j}) \chi_{\{x_{j}\}}(x)\]
	and the \textbf{upper step function of $f$ with respect to $\{x_{0},\dots,x_{n}\}$} as
	\[\phi^{*}(x) = \sum_{j = 1}^{n} M_{j}\chi_{I_{j}}(x) + \sum_{j = 0}^{n} f(x_{j}) \chi_{\{x_{j}\}}(x)\]

	\textbf{Note:} $\phi_{*}(x)$ and $\phi^{*}(x)$ are step functions, and that $\phi_{*}(x) \le f \le \phi^{*}(x)$

	\noindent\rule{\textwidth}{0.2pt}
	Suppose that $g : [a, b] \to \mathbb{R}$ and let $f$ be defined by $f(x) = g(x)$ for $x\in [a,b]$ and $f(x) = 0$ otherwise.
	\begin{enumerate}
	    \setlength\itemsep{0em}
	    \item If $g$ is continuous on $[a, b]$, then $f$ is Riemann-integrable
	    \item If $g$ is a monotone function then $f$ is Riemann-integrable
	\end{enumerate}
\end{thm}

\newpage
\begin{thm}[Dependence on Intervals for Lebesgue]{thm:dependence-on-intervals}{4.8}
	Let $I$ and $J$ be two intervals such that $J \subset I$.
	\begin{enumerate}
	    \item If $f$ is integrable on $I$ then $f$ is also integrable on the subinterval $J$
	    \item If $f$ is integrable on $J$ and simultaneously $f(x) = 0$ for all $x\in I \backslash J$ then $f$ is integrable on $I$ and
			\[\int_{J} f = \int_{I} f\]
		\item If $f$ is integrable on $I$ and $f(x) \ge 0$ for all $x\in I$ then
			\[\int_{J} f \le \int_{I} f\]
		\item Suppose that $I$ can be written as the union of disjoint intervals $I_{n},\, n = 1,2,3,\dots$ and let $f$ be integrable on each of the intervals $I_{n}$. Then $f$ is integrable on $I$ iff
			\[\sum_{n = 1}^{\infty} \int_{I_{n}} \lvert f \rvert < \infty\]
			If this holds, then
			\[\int_{I} f = \sum_{n = 1}^{\infty} \int_{I_{n}} f\]
	\end{enumerate}
\end{thm}


\begin{thm}[Addition of Intervals]{thm:interval-addition}{4.9}
	If any two of these integrals
	\[\int_{a}^{b} f, \quad \int_{b}^{c} f, \quad \int_{a}^{c} f\]
	exist then so does the third and
	\[\int_{a}^{b} f, + \int_{b}^{c} f = \int_{a}^{c} f\]
\end{thm}

\begin{thm}[Fundamental Theorem of Calculus]{thm:differentiability}{4.10}
	Let $I$ be an interval and let $g : I \to \mathbb{R}$ be integrable on $I$. For all $x\in I$ and some fixed $x_{0}\in I$ let $G(x) = \int_{x_{0}}^{x}g$. Suppose $g$ is continuous at $x$ for some $x\in I$ [if $x$ is an endpoint we mean one-sided continuity.] Then $G$ is differentiable at $x$ and $G'(x) = g(x)$. [if $x$ if an endpoint we mean one-sided differentiable]

	\noindent\rule{\textwidth}{0.2pt}
	Suppose $f : I \to \mathbb{R}$ has continuous derivative $f'$ on the interval $I$. Then for any $a, b\in I$:
	\[\int_{a}^{b} f' = f(b) - f(a) \]
\end{thm}

\begin{lma}[Fatoux Lemma]{thm:fatoux}{4.2}
	Let $(f_{n})$ be a sequence of non-negative integrable functions on an interval $I$. Let
	\[f(x) = \liminf_{n\to \infty} f_{n}(x),\quad\text{for all $x\in I$}\]
	If $\liminf_{n\to \infty}\int_{I} f_{n} < \infty$ then $f$ is integrable on $I$ and
	\[\int_{I} f \le \liminf_{n\to \infty}\int_{i} f_{n}\]
\end{lma}

\begin{thm}[Dominated Convergence Theorem]{thm:dominated-convergence-thm}{4.12}
	Let $(f_{n})$ be a sequence of integrable functions on an interval $I$ and assume that
	\[f(x) = \lim_{n\to \infty} f_{n}(x), \quad \text{for all $x\in I$}\].
	Assume also that the sequence $(f_{n})$ is \textbf{dominated} by some integrable function $g$, that is
	\[\lvert f_{n}(x) \rvert \le g(x),\quad \text{for all $x\in I$ and $n = 1,2,\dots,$} \quad \int_{I} g < \infty\]
	Then the function $f$ is integrable on $I$ and
	\[\int_{I} f = \lim_{n\to \infty} \int_{I} f_{n}\]
\end{thm}

\begin{thm}[]{thm:convergent-seq-integration}{4.13}
	Let $(a, b)$ be a bounded interval and suppose that $f_{n} : (a, b) \to \mathbb{R}$ are integrable functions which converges uniformly to a function $f$. Then $f$ is integrable on $(a, b)$ and
	\[\int_{a}^{b} f = \lim_{n\to \infty} \int_{a}^{b} f_{n} \]
\end{thm}

\section{Fourier Series and Orthogonality}



\lipsum[1-12]
\end{multicols}
\end{document}
