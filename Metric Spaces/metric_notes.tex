\documentclass{article}
% \usepackage{showframe}

% \usepackage[dvipsnames]{xcolor}
% custom colour definitions
% \colorlet{colour1}{Red}
% \colorlet{colour2}{Green}
% \colorlet{colour3}{Cerulean}

\usepackage{geometry}
% margins
\geometry{
    a4paper,
    bottom=70pt,
    % margin=70pt
}

\usepackage{graphicx} % Required for inserting images
\usepackage{amsmath}
\usepackage{amsthm}
\usepackage{amsfonts}
\usepackage{amssymb}
\usepackage{preamble}
\usepackage{multicol}
\usepackage{lipsum}
\usepackage{float}
\usepackage[nodisplayskipstretch]{setspace}

% tikz and theorem boxes
\usepackage[framemethod=TikZ]{mdframed}
\usepackage{../thmboxes_v2}
% \usepackage{thmboxes_col}

\usepackage{../coffeestains}


\usepackage{hyperref} % note: this is the final package
\parindent = 0pt
\linespread{1.1}

% Custom Definitions of operators
\DeclareMathOperator{\Ima}{im}
\DeclareMathOperator{\Fix}{Fix}
\DeclareMathOperator{\Orb}{Orb}
\DeclareMathOperator{\Stab}{Stab}
\DeclareMathOperator{\send}{send}
\DeclareMathOperator{\dom}{dom}

\title{Metric Spaces Notes}
\author{Leon Lee}
\renewcommand\labelitemi{\tiny$\bullet$}


\begin{document}

\maketitle
\newpage
\tableofcontents
\newpage

\section{Introduction to Metric Spaces}
\subsection{Defining a Metric}
\textbf{Metric} is another name for distance. A \textbf{Metric Space} is a set equipped with a metric.
A standard example is $\mathbb{R}$ with the standard metric
\[d(x,y) = \lvert x-y \rvert\]
We will now formally define what it means to have a metric



\begin{thm}[Definition of a Metric]{def:metric}{}
    Let $X$ be a non-empty set. A function $d: X \times X \to \mathbb{R} $ is called a \textbf{metric} iff for all $x,\,y,\,z\in X$,
    \begin{itemize}
        \item $d(x,y)\ge 0$ and $d(x,y)=0 \iff x = y$
        \item $d(x,y)=d(y,x)$
        \item $d(x,y)\le d(x,z)+d(z,y)$ (Triangle Inequality)
    \end{itemize}
    A non-empty set $X$ equipped with a metric $d$ is called a \textbf{metric space}
\end{thm}

\subsection{Examples of Metric Spaces}
We can construct a metric space using the \textbf{Absolute value} equipped with the standard triangle inequality
\begin{xmp}[The Real Line]{xmp:real-line}{}
    Let $X = \mathbb{R}$. Define our metric $x: X \times X \to \mathbb{R} $ by
    \[d(x,y)=\lvert x-y \rvert\]
    The first two properties are fairly trivial. The third property follows using the regular triangle inequality
    \[d(x,y) = \lvert x - y \rvert = \lvert (x - z) + (z - y) \rvert \,\le\, \lvert x - z \rvert + \lvert z - y \rvert = d(x,z) + d(z,y)\]
\end{xmp}

\textbf{Remark}: This can be extended not just in $\mathbb{R}^{2}$, but to all $\mathbb{R}^{n}$. By induction,
\[\lvert x_{1} + \cdots + x_{N} \rvert \le \lvert x_{1} \rvert + \cdots + \lvert x_{N} \rvert\]
If $\displaystyle\sum_{n = 1}^{\infty}x_{n}$ converges absolutely, let $N\to +\infty$ to see that
\[\left\lvert \sum_{n = 1}^{\infty}x_{n} \right\rvert \le \sum_{n = 1}^{\infty} \lvert x_{n} \rvert\]


\coffeestainC{0.1}{2}{360}{130 pt}{-100 pt}


\newpage
A second example is the \textbf{Euclidean Plane}. The metric is defined using the \textbf{inner product} and the \textbf{norm}.

\begin{dfn}[Inner Product]{def:inner-product}{}
    The \textbf{inner product} is defined as
    \[\langle x,y \rangle = x_{1}y_{1} + x_{2}y_{2}\]
    Properties of the inner product:
    For all vectors $x,\,y,\,z\in \mathbb{R}^{2}$ and all real scalars $a,\,b$,
    \begin{itemize}
        \item $\langle x,x \rangle\ge 0$ and $\langle x,x \rangle = 0 \iff x = 0$
        \item $\langle x,y \rangle$ = $\langle y,x \rangle$
        \item $\langle ax+by,z \rangle = a\langle x,z \rangle + b\langle y,z \rangle$
    \end{itemize}
\end{dfn}

\textbf{Remark}: This is basically a formalisation of the dot product

\begin{dfn}[Norm]{def:norm}{}
    The \textbf{norm} is defined as:
    \[\lVert x \rVert_{2} = \langle x,x \rangle^{1/2} = \sqrt{x^{2}_{1} + x^{2}_{2}}\]
    Properties of the norm: For all $x,\,y\in\mathbb{R}^{2}$, $a\in\mathbb{R}$
    \begin{itemize}
        \item $\lVert x \rVert_{2} \ge 0$ and $\lVert x \rVert_{2} = 0 \iff x = 0$
        \item $\lVert ax \rVert_{2} = \lvert a \rvert\lVert x \rVert_{2}$
        \item $\lVert x+y \rVert_{2} \le \lVert x \rVert_{2}+\lVert y \rVert_{2}$ (triangle inequality)
    \end{itemize}
\end{dfn}
\textbf{Remark}: This is a formalisation of the "length of a vector"

With these two properties, we can now define the \textbf{Euclidean Metric}
\begin{xmp}[Euclidean Metric]{xmp:euclidean}{}
    For all $x = (x_{1},x_{2}),\, y=(y_{1},y_{2})\in \mathbb{R}^{2}$, define
    \[d_{2}(x,y) = \lVert x-y \rVert_{2}=\sqrt{(x_{1}-y_{1})^{2} + (x_{2}-y_{2})^{2}}\]
\end{xmp}

\textbf{Remark}: Derivation of the triangle inequality is basically the same as Example \ref{xmp:real-line}. 
\[d_{2}(x,y) = \lVert x - y \rVert_{2} = \lVert (x-z) + (z-y) \rVert_{2} \,\le\, \lVert x-z \rVert_{2} + \lVert z-y \rVert_{2} = d_{2}(x,z) + d_{2}(z, y)\]

\subsubsection{Proof of the euclidean triangle inequality}
W.T.S: 
\[\lVert x+y \rVert_{2} \le \lVert x \rVert_{2} + \lVert y \rVert_{2}\]

\textbf{Proof}: Square both sides
\begin{align*}
    \text{LHS}^{2} &= \langle x + y, x + y \rangle  & \text{RHS}^{2}&= \lVert x \rVert^{2}_{2} + \lVert y \rVert^{2}_{2} + 2\lVert x \rVert_{2}\lVert y \rVert_{2}\\
                   &= \langle x,x \rangle + 2\langle x,y \rangle + \langle y,y \rangle \\
                   & = \lVert x \rVert^{2}_{2} + 2\langle x,y \rangle + \lVert y \rVert^{2}_{2}
\end{align*}
\newpage

Discarding the equal terms, we get
\begin{align*}
    \textcolor{red}{\lVert x \rVert^{2}_{2}} + 2\langle x,y \rangle + \textcolor{red}{\lVert y \rVert^{2}_{2}} &\le \textcolor{red}{\lVert x \rVert^{2}_{2}} + \textcolor{red}{\lVert y \rVert^{2}_{2}} + 2\lVert x \rVert_{2}\lVert y \rVert_{2}\\
    \langle x,y \rangle &\le \lVert x \rVert_{2}\lVert y \rVert_{2} \\
    \text{i.e. } x_{1}y_{1} + x_{2}y_{2} &\le \sqrt{x^{2}_{1} + x^{2}_{2}}\sqrt{y^{2}_{1}+y^{2}_{2}}
\end{align*}
This is the \textbf{Cauchy-Schwarz Inequality}. Various ways to prove this (watch lecture 1)

\begin{xmp}[Complex Plane]{xmp:complex-plane}{}
    Let $X = \mathbb{C},\, d: \mathbb{C} \times \mathbb{C} \to \mathbb{R} $
    \[d(z,w) = \lvert z - w \rvert\]
    If $z = a+ib, w = c+id,\, a,b,c,d\in\mathbb{R}$, then
    \[z - w = (a - c) + i(b - d)\]
    therefore,
    \[d(z,w) = \sqrt{(a-c)^{2} + (b-d)^{2}}\]
\end{xmp}

\begin{dfn}[$n$-dimensional Euclidean space]{def:ndim-euclidean-space}{}
    Let $X = \mathbb{R}^{n} = \{(x_{1},x_{2},\dots,x_{n}) : x_{1},x_{2},\dots,x_{n}\in\mathbb{R}\}$
    \newline
    For $x = (x_{1},x_{2},\dots,x_{n})$, $y=(y_{1},y_{2},\dots,y_{n})$ in $\mathbb{R}^{n}$, define
    \[\langle x,y \rangle = x_{1}y_{1} + x_{2}y_{2} + \cdots + x_{n}y_{n} \text{ (inner product)}\]
    \textbf{Properties of $n$-inner product}: For all vectors $x,y,z\in\mathbb{R}^{n}$ and all real scalars $a,b$,
    \begin{itemize}
        \item $\langle x,x \rangle\ge 0$ and $\langle x,x \rangle=0 \iff x = 0$
        \item $\langle x,y \rangle = \langle y,x \rangle$
        \item $\langle ax+by,z \rangle = a\langle x,z \rangle + b\langle y,z \rangle$
    \end{itemize}
    For $x = (x_{1},x_{2},\dots,.x_{n})\in\mathbb{R}^{n}$ define
    \[\lVert x \rVert_{2} = \langle x,x \rangle^{1/2} = \sqrt{x^{2}_{1} + x^{2}_{2} + \cdots + c^{2}_{n}}\text{(norm)}\]
    \textbf{Properties of $n$-norm}: For $x,y\in\mathbb{R}^{n},\,a\in\mathbb{R}$,
    \begin{itemize}
        \item $\lVert x \rVert_{2}\ge 0$ and $\lVert x \rVert_{2}=0 \iff x = 0$
        \item $\lVert ax \rVert_{2}=\lvert a \rvert\lVert x \rVert_{2}$
        \item $\lVert x+y \rVert_{2}\le \lVert x \rVert_{2} + \lVert y \rVert_{2}\text{ (triangle inequality)}$
    \end{itemize}
\end{dfn}

\newpage
\begin{xmp}[Metric in $n$-dim euclidean space]{xmp:ndim-euclidean-space}{}
    For $x = (x_{1},x_{2},\dots,x_{n}),\,y=(y_{1},y_{2},\dots,y_{n})$ in $\mathbb{R}^{n}$, define
    \begin{align*}
        d_{2}(x,y) &= \lVert x - y \rVert_{2} \\
                   &= \sqrt{(x_{1}-y_{1})^{2} + (x_{2}-y_{2})^{2} + \cdots + (x_{n} - y_{n})^{2}}
    \end{align*}
    Triangle inequality, cauchy schwarz, yadda yadda same as $2$-dim case
\end{xmp}

\subsubsection{L space}

For two sequences $x = (x_{1}, \dots ,x_{n},\dots),\, y=(y_{1},\dots,y_{n},\dots)$ of real numbers we wish to define
\[d_{1}(x,y) = \sum_{n = 0}^{\infty} \lvert x_{n} - y_{n} \rvert\]
We need this series to converge - in particular when $y=(0,\dots,0,\dots)$, we need the series $\displaystyle\sum_{n = 1}^{\infty}\lvert x_{n} \rvert$ to converge

\begin{dfn}[l space]{l-space}{}
    We denote by $\ell^{1}$ the set of real sequences $(x_{n})_{n\in\mathbb{N}}$ for which the series $\displaystyle\sum_{n = 1}^{\infty}\lvert x_{n} \rvert$ converges.
\end{dfn}

If $x,y\in \ell^{1}$ i.e. if $\displaystyle\sum_{n = 1}^{\infty}\lvert x_{n} \rvert$ and $\displaystyle\sum_{n = 1}^{\infty}\lvert y_{n} \rvert$ converge, then $\displaystyle\sum_{n = 1}^{\infty} \lvert x_{n} - y_{n} \rvert$ converges, because for all $n$,
\[\lvert x_{n} - y_{n} \rvert \le \lvert x_{n} \rvert + y_{n}\]

For $x = (x_{1}, \dots, x_{n}, \dots)$ in $\ell^{1}$, we may now define
\[\lVert x \rVert_{1} = \sum_{n = 1}^{\infty}\lvert x_{n} \rvert\]
For $x = (x_{1}, \dots , x_{n}, \dots),\, y = (y_{1}, \dots, y_{n}, \dots)$ in $\ell^{1}$ we may now define
\[d_{1}(x,y) = \lVert x - y \rVert_{1} = \sum_{n = 1}^{\infty} \lvert x_{n} - y_{n} \rvert\]
\newpage
\subsection{Real Vector Spaces}
\begin{dfn}[Real Vector Spaces]{r-vector-spaces}{}
    A \textit{real vector space} is a set $X$ with two operations, addition($+$) and scalar multiplication $\cdot$, with the following properties: for all $x,y,z\in X$, $a,b\in\mathbb{R}$, we have $x + y, a\cdot x\in X$, and
    \begin{itemize}
        \item $x + y = y + x$
        \item $x + (y + z) = (x + y) + z$
        \item There is an element of $X$ denoted by $0$ such that, for all $x$, $0 + x = x + 0 = x$
        \item For every $x\in X$ there exists an element of $X$ denoted by $-x$ such that $x + (-x) = (-x) + x = 0$
        \item $a \cdot (x + y) = a \cdot x + a \cdot y$
        \item $(a + b) \cdot x = a \cdot x + b \cdot x$
        \item $a \cdot (b \cdot x) = (ab) \cdot X$
        \item $1 \cdot x = x$
    \end{itemize}
    (we usually write $ax$ instead of $x$)
\end{dfn}

\subsubsection{Normalising l 1}
Properties: For all sequences $x,y\in \ell^{1}$ and all real scalars $a$,
\begin{itemize}
    \item $\lVert x \rVert_{1}\ge 0$ and $\lVert x \rVert_{1} = 0 \iff x = 0$
    \item $\lVert ax \rVert_{1} = \lvert a \rvert \lVert x \rVert_{1}$
    \item $\lVert x + y \rVert_{1} \le \lVert x \rVert_{1} + \lVert y \rVert_{1}$
\end{itemize}

\subsubsection{Space l-2}
We denote by $\ell^{2}$ the set of real sequences $(x_{1}, \dots ,x_{n},\dots)$ such that the seriese $\displaystyle \sum_{n = 1}^{\infty} \lvert x_{n} \rvert^{2}$ converges
For $x = (x_{1}, \dots, x_{n}, \dots)\in \ell^{2}$, $y = (y_{1},\dots,y_{n},\dots)\in\ell^{2}$ we define
\begin{itemize}
    \item $\langle x,y \rangle = \sum_{n = 1}^{\infty}x_{n}y_{n}$ (inner product)
    \item $\lVert x \rVert_{2} = \displaystyle\left(\sum_{n = 1}^{\infty}\lvert x_{n} \rvert^{2}\right)^{1/2}$ (norm)
    \item $\displaystyle d_{2}(x,y) = \lVert x - y \rVert_{2} = \left(\sum_{n = 1}^{\infty} \lvert x_{n} - y_{n} \rvert^{2}\right)^{1/2}$ (Metric)
\end{itemize}

\begin{thm}[4]{thm:l2-vs}{}
    $\ell^{2}$ is a real vector space
    proof icba
\end{thm}

more stuff on $\ell^{2}$ - typical properties watch video 1

\subsection{Generalising metric space features}

\begin{dfn}[Normed Vector Spaces]{dfn:normed-vec-spaces}{}
    A \textit{normed vector space} (or \textit{normed linear space} or \textit{normed space}) is a real vector space $X$ equipped with a \textit{norm}, i.e. a function that assigns to every vector $x\in X$ a real number $\lVert x \rVert$ so that, for all vectors $x$ and $y$ in $X$ and all real scalars $a$,
    \begin{itemize}
        \item $\lVert x\rVert\ge 0 $ and $\lVert x \rVert = 0 \iff x = 0$
        \item $\lVert ax \rVert = \lvert a \rvert\lVert x \rVert$
        \item $\lVert x+y \rVert\le \lVert x \rVert + \lVert y \rVert$
    \end{itemize}
    If $(X, \lVert \cdot \rVert)$ is a normed vector space then
    \[d(x,y) = \lVert x - y \rVert\]
    defines a metric in $X$
\end{dfn}

\begin{dfn}[Inner Product Spaces]{dfn:inner-product-space}{}
    Let $X$ be a real vector space. An \textit{inner product} on $X$ is a function that assigns to every pair $(x,y)\in X \times X $ a real number denoted by $\langle x,y \rangle$ and has the following properties
    \begin{itemize}
        \item $\langle x,x \rangle\ge 0$ and $\langle x,x \rangle = 0 \iff x = 0$
        \item $\langle x,y \rangle = \langle y,x \rangle$
        \item $ax + by, z = a\langle x,z \rangle + b\langle y,z \rangle$
    \end{itemize}
    A \textit{real inner product space} is a real vector space equipped with an inner product.
    If $\lVert \cdot , \cdot \rVert$ is an inner product on $X$, then
    \[\lVert x \rVert = \sqrt{\langle x,x \rangle}\]
    defines a norm and
    \[d(x,y) = \lVert x - y \rVert\]
    defines a metric
\end{dfn}

\begin{xmp}[Discrete metric]{xmp:discrete-metric}{}
    Let $X$ be a non-empty set. Define $d: X \times X \to \mathbb{R}$ by
    \[d(x,y) = \begin{cases}
        0, &x = y \\
        1, x \ne y
    \end{cases}\]
    Example of metric space without norm or inner prod. Another example is post office metric
\end{xmp}

theres lots of examples, i kinda cba

\newpage

\subsection{Open Balls}

\begin{dfn}[Open Ball]{def:open-ball}{}
    Let $(X, d)$ be a metric space, $c$ be a point in $X$, and $r > 0$. The \textbf{open ball} with center $c$ and radius $r$ is defined by
    \[B(c,r) = \{x\in X: d(c,x) < r\}\]
\end{dfn}

Note: there are lots of different notations for this, e.g. calling it a sphere

\textbf{Example:} on the real line with the standard metric
\[b(c,r) = \{x\in\mathbb{R} : \lvert x - c \rvert < r\} = (c - r, c + r)\]

\textbf{Example:} on the real plane with the Euclidean metric, $X = \mathbb{R}^{2}$m
\[d_{2}(x,y) = \sqrt{\lvert x_{1} - y_{1} \rvert^{2} + \lvert x_{2} - y_{2} \rvert}^{2}\]
$B(c,r)$ is the open disc with center $c$ and radius $r$

\noindent\rule{\textwidth}{0.2pt}

Watch lecture recording for examples of open balls on:
\begin{itemize}
    \item Discrete metric
    \item $\mathbb{R}^{2}$ with the $d_{1}$ metric
    \item $\mathbb{R}^{2}$ with the $d_{\infty}$ metric
\end{itemize}

\newpage

\section{Convergence}

\subsection{Convergent Sequences in Metric Spaces}
On the real line, $x_{n}\to x$ iff for every positive $\epsilon$, there exists an index $N$ such that for all indices $n$ where $n\ge N$, we have $\lvert x_{n} - x \rvert < \epsilon$.

\begin{dfn}[Convergent Sequence]{metric-convergence}{}
    Let $(X,d)$ be a metric space, $(x_{n})^{\infty}_{n=1}$ be a sequence in $X$, and $x\in X$. We say that $(x_{n})^{\infty}_{n=1}$ converges to $x$ iff for every positive $\epsilon$, there exists an index $N$ s.t. for all indices $n$ with $n\ge N$a we have $d(x_{n}, x) < \epsilon$.

    Observe that:
    \begin{itemize}
        \item $d(x_{n}, x)< \epsilon$ is equivalent to $x_{n}\in B(x,\epsilon)$.
        \item $x_{n}\to x$ in $(X,d)$ iff $d(x_{n}, x)\to 0$ on the real line
    \end{itemize}
\end{dfn}

\begin{thm}[Uniqueness of metric limit]{thm:uniqueness-metric-limit}{}
    \begin{itemize}
        \item Let $(X,d)$ be a metric space, and $x,x'\in X,\,x\ne x'$. Then there exists a positive radius $r$ s.t. $B(x,r)\cap B(x',r) = \emptyset$
        \item A sequence in a metric space can have at most one limit
    \end{itemize}
\end{thm}

\textbf{Proof of first:} $d(x,x') > 0$ because $x \ne x'$. Choose any $r$ with $0  < r \le \frac{d(x,x')}{2}$. If $y\in B(x,r)$, then $d(y,x)<r$, therefore
\[d(y,x' \ge d(x,x') - d(y,x) > d(x,x') - r)\]
and $d(x,x') - r \ge r$, therefore
\[d(y,x') > r\]
Therefore, $y\not\in B(x', r)$

\textbf{Proof of second: } Let $x_{n} \to x$ and $x_{n}\to x'$ in a metric space $(X,d)$. We claim that $x = x'$.
Assume $x\ne x'$. Let $r > 0$ be s.t.
\[B(x,r) \cap B(x',r) = \emptyset\]
Since $x_{n}\to x$, there exists $N$ s.t. for all $n$ with $n \ge N$ we have
\[x_{n} \in B(x,r)\]
Since $x_{n}\to x$, there exists $N'$ s.t. for all $n$ with $n \ge N'$ we have
\[x_{n} \in B(x',r)\]
For any $n$ with $n \ge \max \{N, N'\}$, the term $x_{n}$ belongs to both balls - contradiction

\newpage
\begin{xmp}[convergence in ($\mathbb{R}^N, d_2$)]{xmp:convergence}{}
    A sequence
    \begin{align*}
        x_{1} &= (x_{11}, \dots,x_{1j}, \dots x_{1N}) \\
        x_{2} &= (x_{21}, \dots,x_{2j}, \dots x_{2N}) \\
        \vdots & \\
        x_{n} &= (x_{n1}, \dots,x_{nj}, \dots x_{nN}) \\
        \vdots \\
        \downarrow \\
        x &= (x_{1}, \dots, x_{j}, \dots, x_{N})
    \end{align*}
    in $\mathbb{R}^{N}, d_{2}$ converges to $x = (x_{1}, \dots, x_{j}, \dots, x_{N})$ iff for each $j$,
    \[x_{nj}\xrightarrow[j\to+\infty]{} x_{j}\]
\end{xmp}

\noindent\rule{\textwidth}{0.2pt}

Watch lecture recording 23/01 for examples of:
\begin{itemize}
    \item Convergence in $\ell^{2}$
    \item Convergecnce in $C([a,b])$
\end{itemize}


\begin{dfn}[Bounded Sequence]{def:bounded-sequence}{}
    A sequence in a metric space is said to be \textbf{bounded} iff there exists an open ball that contains all of its terms
\end{dfn}

Note: this is the same definition as "sequence is bounded if there is upper and lower bound", as open ball implies the same thing

\begin{thm}[]{thm:convergent-seq-bound}{}
    Every convergence is bounded
\end{thm}

\textbf{Proof:} Let $x_{n}\to x$ in a metric space $(X,d)$. There exists an index $N$ s.t. for all $n$ with $n\ge N$,
\[x_{n}\in B(x,1)\]
Let $r$ be any positive number such that
\[r>1,\,r>d(x,x_{1}),\dots,r>d(x,x_{N-1})\]
Then, for all $n$,
\[d(x_{n}, x) < r\]
therefore
\[x_{n}\in B(x,r)\]

\subsection{Cauchy Sequences}
Convergence: For every $\epsilon$, there is an $N$ such that for $n\ge N$, $d(x_{n}, x) < \epsilon$
\[x_{1}\quad x_{2} \quad \cdots \quad x_{N} \quad \cdots x_{n} \quad \cdots \quad \to x\]
Replace $x$ by any $x_{m}$ with $m \ge N$
\[x_{1}\quad x_{2} \quad \cdots \quad x_{N} \quad \cdots x_{n} \quad \cdots \quad x_{m} \quad \cdots\]
'$d(x_{n}, x) < \epsilon$' becomes '$\forall m \ge N,\,d(x_{n}, x_{m}) < \epsilon$'

\begin{dfn}[Cauchy Sequence]{def:cauchy-seq}{}
    A sequence $(x_{n})^{\infty}_{n=1}$ in a metric space $(X,d)$ is said to be a \textbf{Cauchy sequence} iff for every positive $\epsilon$, there exists an index $N$, s.t. for all indices $n,m$ with $n,m\ge N$,
    \[d(x_{n}, x_{m}) < \epsilon\]
\end{dfn}

\begin{thm}[]{thm:convergent-to-cauchy}{}
    If a sequence in a metric space converges, then it is a Cauchy sequence
\end{thm}

\textbf{Proof: } If $x_{n}\to L$ in a metric space $(X,d)$, then for every positive $\epsilon$, there exists an index $N$, such that for all indices $n$ with $n\ge N, d(x_{n}, L) < \frac{\epsilon}{2}$. Therefore for all $n,m\ge N$,
\[d(x_{n}, x_{m})\le d(x_{n}, L) + d(x_{m}, L) < \frac{\epsilon}{2} +  \frac{\epsilon}{2} = \epsilon\]

Note: The converse is not true.

\textbf{Counterexample: }
\[X = (0,1), d(x,y) = \lvert x - y \rvert,\,x_{n} = \frac{1}{n},\,(n\ge 2)\]
This sequence is Cauchy but not convergent

Cauchy: Let $\epsilon$ be positive. Pick $N$ s.t. $\frac{1}{N} < \frac{\epsilon}{2}$. For $n,m\ge N$ we have
\[d(x_{n}, x_{m}) = \left\lvert \frac{1}{n} - \frac{1}{m} \right\rvert \le \frac{1}{n} + \frac{1}{m} \le \frac{2}{N} < \epsilon\]

Not convergent: Let $x\in (0,1)$. Find $N$ s.t. $\frac{1}{N}< x$. For $n\ge N$ we have $x_{n} = \frac{1}{n} \le \frac{1}{N}$, so the open interval $(\frac{1}{N}, 1)$ contains $x$ and only finitely many terms of the sequence. Therefore $x_{n}\not\to x$

\noindent\rule{\textwidth}{0.2pt}
Watch Lecture 23/01 for example of counterexample
\begin{itemize}
    \item Metric spaces $(\mathbb{R}, d_{\mathbb{R}})$ and $(\mathbb{Q}, d_{\mathbb{Q}})$
\end{itemize}

\begin{dfn}[Complete Metric Spaces]{def:complete-ms}{}
    A metric space is said to be \textbf{complete} if and only if every Cauchy Sequence is convergent
\end{dfn}

\textbf{Examples:}
\begin{itemize}
    \item $\mathbb{R}$ with the standard metric is complete
    \item $\mathbb{Q}$ with the standard metric is not complete
    \item $(0,1)$ with the standard metric is not complete
    \item $[0,1]$ with the standard metric is complete
    \item $\mathbb{R}^{n},\,\ell^{p},\,C([a,b])$ is complete (proof later)
\end{itemize} 

\subsection{Open sets and closed sets}

\begin{dfn}[Open Sets and Closed Sets]{def:open-closed-set}{}
    Let $(X,d)$ be a metric space.
    \begin{itemize}
        \item A subset $G$ of $X$ is said to be \textbf{open} iff for every point $x$ in $G$ there exists a positive radius $r$ such that $B(x,r)\subseteq G$.
        \item A subset $F$ of $X$ is said to be \textbf{closed} iff $F^{c}$ is open
    \end{itemize}
\end{dfn}

\textbf{Example:} In any metric space $(X, d)$, the sets $\emptyset$ and $X$ are both open and closed.

$\emptyset$ is open because the following statement is true:
\[\forall x (x\in\emptyset \implies \exists r \dots)\]
$X$ is open because, for every $x$ in $X$ we can take $r = 1234$ to have $B(x,r)\subseteq X$

$\emptyset^{c} = X$ and $X^{c} = \emptyset$ are closed

\noindent\rule{\textwidth}{0.2pt}
Watch lecture recording 26/01 for details on examples
\begin{itemize}
    \item Every open ball is an open set
    \item If $d$ is the discrete metric on a non-empty set $X$, then every subset of $X$ is both open and closed
    \item $X = \mathbb{Z},\,d(x,y) = \lvert x - y \rvert$, all subsets of $X$ are both open and closed
\end{itemize}

\begin{dfn}[Discrete Metric Space]{def:discrete-ms}{}
    A metric space is called \textbf{discrete} iff all its subsets are open (equiv. all subsets are closed)
\end{dfn}

\textbf{Example:} $[0,1]\cap (2,3)$

\begin{thm}[Properties of open sets]{thm:open-set-props}{}
    Let $(X,d)$ be a metric space
    \begin{enumerate}
        \item The union of any family of open sets is an open set
        \item The intersection of finitely many open sets is an open set
    \end{enumerate}
\end{thm}

\textbf{Proof for 1:} Let $(G_{i})_{i\in I}$ be a family of open sets and define $G = \bigcup_{i\in I} G_{i}$. If $x\in G$, then $x\in G_{i}$ for some $i$. Since $G_{i}$ is open, there exists a positive $r$ such that $B(x,r)\subseteq G_{i}$. Then $B(x,r)\subseteq G$


\textbf{Proof for 2:} Let $G_{1}, \dots, G_{n}$ be open sets. Define $G = G_{1} \cap \cdots \cap G_{n}$. If $x\in G$, then $x\in G_{i}$ for all $i$. Since each $G_{i}$ is open, there exists a positive $r_{i}$ such that $B(x,r_{i})\subseteq G_{i}$. Let $r = \min \{r_{1},\dots,r_{n}\}$. For each $i$,
\[B(x,r)\subseteq B(x,r_{i})\subseteq G_{i}\]
Therefore, $B(x,r)\subseteq G_{1} \cap \cdots \cap G_{n} = G$

\newpage
\begin{thm}[Infinite open sets]{thm:infinite-open-sets}{}
    The intersection of infinitely many open sets is not always an open set
    
    For example, let $G_{n} = (- \frac{1}{n}, \frac{1}{n}), n = 1,2,\dots$ on the real line with the standard metric.

    Each $G_{n}$ is open but
    \[\bigcap\limits_{n = 1}^{\infty} G_{n} = \{0\}\]
\end{thm}


\begin{thm}[Relatively open sets]{thm:relatively-open}{}
    Let $(X,d)$ be a metric space and $A$ be a non-empty subset of $X$ equipped with the induced metric $d_{A}$. Let $G\subseteq A$. $G$ is open in $(A, d_{A})$ iff there exists a subset $O$ of $X$, open in $(X,d)$, such that $G = A \cap O$

    The open sets of $(A, d_{A})$ are sometimes referred to as \textbf{relatively open}
\end{thm}

\begin{thm}[]{thm:open-set-converge}{}
    Let $(X, d)$ be a metric space, $(x_{n})^{\infty}_{n=1}$ be a sequence in $X$ and $x$ be a point in $X$.

    $x_{n}\to x$ iff every open set that contains $x$ contains eventually all terms of the sequence
\end{thm}

\textbf{Proof:} Assume $x_{n}\to x$. Let $G$ be any open set with $x\in G$. There is a positive $r$ such that $B(x,r) \subseteq G$. There is an $N$ such that for all $n$ with $n \ge N$ we have $x_{n}\in B(x,r)$, hence, $x_{n}\in G$.

Conversely, assume that every open set containing $x$ contains eventually all terms of the sequence. Every open ball centered at $x$ is an open set, therefore it contains eventually all terms of the sequence. It follows that $x_{n}\to x$.


\begin{dfn}[Neighbourhoods of points]{def:neighbourhood}{}
    An \textbf{open neighbourhood} of a point $x$ is any open set that contains $x$. $x_{n}\to x$ iff every open neighbourhood of $x$ contains eventually all terms of the sequence.

    \noindent\rule{\textwidth}{0.2pt}

    A \textbf{neighbourhood} of a point $x$ is a set that contains an open neighbourhood of $x$.  $x_{n}\to x$ iff every neighbourhood of $x$ contains eventually all terms of the sequence.
\end{dfn}

\begin{thm}[Properties of Closed sets]{thm:closed-set-props}{}
    Let $(X, d)$ be a metric space.
    \begin{enumerate}
        \item The intersection of any family of closed sets is a closed set
        \item The union of finitely many closed sets is a closed set.
    \end{enumerate}
\end{thm}

\textbf{Proof for 1:} Let $(F_{i})_{i\in I}$ be a family of closed sets. Then each $F_{i}^{c}$ is open, therefore, $\displaystyle\bigcup_{i\in I} F^{c}_{i}$ is open, therefore $\displaystyle\left(\bigcup_{i\in I} F^{c}_{i}\right)$ is closed. By De Morgan's rule, $\displaystyle\left(\bigcup_{i\in I} F^{c}_{i}\right)^{c} = \bigcap_{i\in I} F_{i}$. Therefore, $\displaystyle\bigcap_{i\in I} F_{i}$ is closed.

\textbf{Proof for 2}: Let $F_{1},\dots, F_{n}$ be closed sets. Then $F^{c}_{1},..,F^{c}_{n}$ are open, therefore $F^{c}_{1} \cap \cdots \cap F^{c}_{n}$ is open, therefore $(F^{c}_{1} \cap \cdots \cap F^{c}_{n})^{c}$ is closed. By de Morgan's rule, $(F^{c}_{1}\cap \cdots\cap F^{c}_{n})^{c} = F \cup \cdots \cup F_{n}$. Therefore, $F \cup \cdots \cup F_{n}$ is closed

\begin{thm}[Infinite closed sets]{thm:infinite-closed-sets}{}
    The union of infinitely many closed sets is not always a closed set.

    For example, let $F_{n} = [\frac{1}{n}, 1], n=1,2,\dots,$ on the real line with the standard metric.
    Each $F_{n}$ is closed but
    \[\bigcup\limits_{n = 1}^{\infty}F_{n} = (0,1]\]
    is not closed.
\end{thm}

\noindent\rule{\textwidth}{0.2pt}
Watch lecture recording 30/01 for examples


\begin{thm}[]{thm:closed-ms-convergence}{}
    A subset $F$ of a metric space is closed iff the limit of every convergent sequence of elements of $F$ belongs to $F$
\end{thm}

\textbf{Proof $\implies$}: Assume $F$ is closed, and let $(x_{n})^{\infty}_{n=1}$ be a convergent sequence of elements of $F$. Let $x$ be its limit. We wish to show that $x\in F$. We argue by contradiction. Suppose $x\not\in F$. Then $x\in F^{c}$, and since $F^{c}$ is open, there exists a positive $r$ such that $B(x,r) \subseteq F^{c}$. Then $B(x,r)$ contains no terms of the sequence - contradiction

\textbf{Proof $\impliedby$}: assume that the limit of every convergent sequence of elements of $F$ belongs to $F$. We wish to show that $F$ is closed.

We show that $F^{c}$ is open. Let $x\in F^{c}$. We need to show that there exists a postive $r$ such that $B(x,r)\subseteq F^{c}$. If not, then for every $r$ there exists a point in $B(x,r)$ that belongs to $F$.

Using this with $r = \frac{1}{n}, n=1,2,3,\dots$, we find points $x_{n}$ with $x_{n}\in B(x, 1 / n)$ and $x_{n}\in F$. Then $x_{n}\to x$ but $x\not\in F$a. Contradiction

\noindent\rule{\textwidth}{0.2pt}
Watch lecture recording 30/01 for examples

\begin{itemize}
    \item In any metric space $(X,d)$, singletons $F = \{x\}$ are closed.
    \item In any metric space, any finite set is closed because
        \[\{x_{1},\dots,x_{n}\} = \{x_{1}\}\cup \cdots \cup \{x_{n}\}\]
\end{itemize}


\subsection{Closure}

\begin{dfn}[Closure]{def:closure}{}
    Let $(X, d)$ be a metric space and $A \subseteq X$. The \textbf{closure} of $A$, deonted by $\overline{A}$, is the smallest closed subset of $X$ that contains $A$

    There exists at least one closed subset of $X$ that contains $A$, namely $X$ itself. The smallest closed subset of $X$ that contains $A$ is
    \[\bigcap\limits_{A \subseteq F \subseteq X,\,F \text{closed}} F\]
\end{dfn}

\newpage
\begin{thm}[Properties of Closure]{thm:closure-props}{}
    Let $(X, d)$ be a metric space and $A,\,B \subseteq X$.
    \begin{enumerate}
        \item $\overline{\emptyset} = \emptyset$ and $\overline{X} = X$
        \item $A \subseteq \overline{A}$ and $\overline{A}$ is closed
        \item $A$ is closed iff $A = \overline{A}$
        \item $\overline{\overline{A}} = \overline{A}$
        \item If $A \subseteq B$, then $\overline{A} \subseteq \overline{B}$
        \item $\overline{A \cup B} = \overline{A} \cup \overline{B}$
    \end{enumerate}
\end{thm}

Lecture 30/01 45m for proofs

\noindent\rule{\textwidth}{0.2pt}
\textbf{Example}: $X = \mathbb{R},\,d(x,y) = \lvert x - y \rvert,\,A = (0,1)$. We claim that $\overline{A} = [0,1]$

$A \subseteq [0,1]$ and $[0,1]$ is a closed set. The smallest such set is $\overline{A}$. Therefore $\overline{A} \subseteq [0,1]$. 

Next we show that $[0,1] \subseteq \overline{A}$. clearly, $(0,1) = A \subseteq \overline{A}$

$(1 / 2, 1 /3 ,\dots, 1 /n \dots)\to 0$, each term belongs to $\overline{A}$, and $\overline{A}$ is closed, therefore $0\in \overline{A}$. Similarly, $1\in \overline{A}$

\noindent\rule{\textwidth}{0.2pt}
Watch lecture recording 02/02 10m for more in-depth examples of closure things
\begin{itemize}
    \item On the real line with the standard metric, $\overline{(a,b)} = [a,b]$
    \item In $\mathbb{R}^{n}$ with the Euclidean metric $d_{2}$, the closure of the open ball $B(c,r)$ is the closed ball $\{x\in\mathbb{R}^{n}: d_{2}(x,c)\le r\}$
    \item On the complex plane with its standard metric, the closure of an open disc is the corresponding closed disc
    \item Let $X$ be a non-empty set with the discrete metric, $c\in X$ and $r =1$. Then $B(c,1) = \{c\}$, therefore $\overline{B(c,1)} - \overline{\{c\}} = \{c\}$, while
        \[\{x\in X : d(x,c) \le 1\} = X\]
        The closure of an open ball is not always equal to the corresponding closed ball
    \item $X = \mathbb{R},\,d(x,y) = \lvert x - y \rvert$. $\overline{\mathbb{Q}} = \mathbb{R}$
\end{itemize}

\begin{dfn}[Dense Subset of a Metric Space]{def:dense-subset}{}
    Let $(X, d)$ be a metric space. A subset $D$ of $X$ is said to be \textbf{dense} iff $\overline{D} = X$
\end{dfn}

Random fact: In $\mathbb{R}^{n}$ with the Euclidean metric $d_{2}$, $\mathbb{Q}^{n}$ is dense.

\newpage
\begin{thm}[Closure Equivalence]{thm:closure-equivalence}{}
    Let $(X, d)$ be a metric space, $A \subseteq X, x\in X$. The following are equivalent
    \begin{enumerate}
        \item $x\in\overline{A}$
        \item For every positive $r$, $B(x,r) \cap A \ne \emptyset$
        \item There exists a sequence $(a_{n})_{n\in \mathbb{N}}$ with $a_{n}\in A$ for all $n$, such that $a_{n}\to x$
    \end{enumerate}
    A point $x$ with any of these properties is called an \textbf{adherent point} of $A$. So, $\overline{A}$ is the set of all adherent points of $A$.
\end{thm}

\textbf{Example}: $X = \mathbb{R},\,d(x,y) = \lvert x - y \rvert, A = (0,1)\cup \{2\},\, \overline{A} = [0,1] \cup \{2\}$

$2$ is an adherent point of $A$. $0$ is an adherent point of $A$.

Observe: $2\in A, 0\not\in A$

\noindent\rule{\textwidth}{0.2pt}
\textbf{Proof}: $1 \implies 2$

Assume $x\in \overline{A}$. Fix a positive $r$. We show: $B(x,r)\cap A\ne \emptyset$.

The set $\overline{A} \backslash B(x,r)$ is closed and $\overline{A} \backslash B(x,r) \subsetneq \overline{A}$

Therefore, $A \not \subseteq \overline{A} \backslash B(x,r)$

Therefore there exists an element $a\in A$ s.t. $a\not\in \overline{A} \backslash B(x,r)$. But $a\in \overline{A}$. Therefore $a\in B(x,r)$

\textbf{Proof}: $2 \implies 3$

If $A$ intersects every open ball centered at $x$, then for every $n$ there is a point $a_{n}$ that belongs to $A$ and to $B(x, 1 /n)$. Then $d(a_{n}, x) < 1 /n$, therefore $a_{n}\to x$

\textbf{Proof}: $3\implies 1$
Assume that there is a sequence $(a_{n})^{\infty}_{n=1}$ such that $a_{n}\in A$ for all $n$, and $a_{n}\to x$. We show that $x\in \overline{A}$.

For each $n$ we have $a_{n}\in \overline{A}$. Also, $a_{n}\to x$ and $\overline{A}$ is closed. Therefore $x\in \overline{A}$

\begin{dfn}[Limit points of sets]{def:limit-point}{}
    Let $(X, d)$ be a metric space, $A \subseteq X$ and $x\in X$. We say that $x$ is a \textbf{limit point} or an \textbf{accumulation point} of $A$ iff every open ball centered at $x$ contains an element of $A$ distinct from $x$, i.e.
    \[\forall r > 0 \quad (B(x,r) \backslash \{x\}) \cap A \ne \emptyset\]
    The set of all limit points of $A$ is called the \textbf{derived set} of $A$ and is denoted by $A'$ or $\tilde{A}$.
\end{dfn}

\textbf{Note w/o proof}: $x$ is a limit point of $A$ iff there exists a sequence $(a_{n})^{\infty}_{n=1}$ such that $a_{n}\in A, a_{n} \ne x$ for all $n$, and $a_{n}\to x$

\textbf{Note w/o proof}: Let $(X, d)$ be a metric space and $A \subseteq X$. Then $\overline{A} = A \cup A'$

\noindent\rule{\textwidth}{0.2pt}
\textbf{Example}: On the real line with the standard metric, let $A = (0,1) \cup \{2\}$. Then $\overline{A} = [0,1]\cup \{2\}$, so $0,2\in \overline{A}$
$0$ is a limit point of $A$
$2$ isn't a limit point of $A$

\newpage
\subsection{Continuous functions between metric spaces}

\begin{dfn}[Continuity at a point]{def:metric-point-continuity}{}
    Let $(X, d_{X}),\, (Y,d_{Y})$ be metric spaces and $f: X \to Y $ be a function. We say that $f$ is \textbf{continuous at a point} $x_{0}$ in $X$ iff for for every positive $\epsilon$, there exists a positive $\delta$, s.t., for all $x\in X$ with $d_{X}(x,x_{0}) < \delta$ we have $d_{Y}(f(x), f(x_{0})) < \epsilon$
    
    \noindent\rule{\textwidth}{0.2pt}
    Alternatively, $f$ is \textbf{continuous at a point} $x_{0}\in X$ iff, for every positive $\epsilon$, there exists a positive $\delta$, such that, for all $x\in B_{X}(x_{0}, \delta)$ we have $f(x)\in B_{Y}(f(x_{0}), \epsilon)$
\end{dfn}

\begin{dfn}[Continuity of a function]{def:metric-continuity}{}
    Let $(X, d_{X}), (Y, d_{Y})$ be metric spaces. A function $f: X \to Y $ is said to be \textbf{continuous} iff it is continuous at every point in $X$
\end{dfn}

\textbf{Example}: Let $(X, d)$ be a metric space and $p$ be a point in $X$. Define $f: X \to \mathbb{R} $ by $f(x) = d(x,p)$. $f$ is continuous.

Watch lecture recording 02/02 40m for proof

\begin{thm}[]{}{}
    Let $(X,d_{X}), (Y, d_{Y})$ be metric spaces, $f : X \to Y$ be a function and $x_{0}$ be a point in $X$. Then $f$ is continuous at $x_{0}$ iff for every open neighbourhood $G$ of $f(x_{0})$ there exists an open neighbourhood $O$ of $x_{0}$ such that, for all $x\in O$, we have $f(x) \in G$
\end{thm}

\begin{proof}
    Assume $f$ is continuous at $x_{0}$. Let $G$ be an open set in $Y$ with $f(x_{0})\in G$. There exists a positive $\epsilon$ such that $B_{Y}(f(x_{0}), \epsilon) \subseteq G$. By continuity, there exists a positive $\delta$ such that for all $x\in B_{X}(x_{0}, \delta)$ we have $f(x)\in B_{Y}(f(x_{0}), \epsilon)$. Let $O = B_{X}(x_{0}, \delta)$. For all $x\in O$ we have $f(x)\in G$
\end{proof}

\noindent\rule{\textwidth}{0.2pt}
Conversely, assume that for every open neighbourhood $G$ of $f(x_{0})$ there exists an open neighbourhood $O$ of $x_{0}$ s.t. for all $x\in O$, we have $f(x)\in G$. We wish to show that $f$ is continuous at $x_{0}$

Let $\epsilon$ be positive. Apply our hypothesis with $G = B_{Y}(f(x_{0}), \epsilon)$ to see that there exists an open set $O$ in $X$ with $x_{0}\in O$, s.t. for all $x\in O$ we have $f(x)\in G$.

Since $O$ is open, there exists a positive $\delta$ such that $B_{X}(x_{0}, \delta) \subseteq O$.

For all $x$ in $B_{X}(x_{0}, \delta)$ we have $f(x) \in B_{Y}(f(x_{0}), \epsilon)$

\begin{thm}[Continuity and Convergence]{thm:continuity-conv-equivalence}{}
    Let $(X, d_{X}),\, (Y, d_{Y})$ be metric spaces, $x_{0}$ be a point in $X$, and $f : X \to Y$ be a function. The following are equivalent:
    \begin{enumerate}
        \item $f$ is continuous at $x_{0}$
        \item For every sequence $(x_{n})^{\infty}_{n = 1}$ in $X$, if $x_{n}\xrightarrow[n\to +\infty]{}$ in $(X, d_{X})$, then $f(x_{n}) \xrightarrow[n\to +\infty]{} f(x_{0})$ in $(Y, d_{Y})$
    \end{enumerate}
\end{thm}

\newpage
\begin{proof}
    $1 \implies 2$: Assume $f$ is continuous at $x_{0}$ and let $x_{n}\to x_{0}\in X$

Let $\epsilon$ be positive. There exists a positive $\delta$ such that, for all $x\in B_{X}(x_{0},\delta),\,f(x)\in B_{Y}(f(x_{0}), \epsilon)$. Eventually all $x_{n}$ belong to $B_{X}(x_{0}, \delta)$. Therefore eventually all $f(x_{n})$ belong to $B_{Y}(f(x_{0}), \epsilon)$
\noindent\rule{\textwidth}{0.2pt}
$2 \implies 1$: Contrapositive - not $1 \implies$ not 2

Assume that $f$ is not continuous at $x_{0}$. Then
\[\text{not}\left( \forall \epsilon,\,\exists \delta,\,\forall x\in B_{X}(x_{0}, \delta) \quad f(x)\in B_{Y}(f(x_{0}), \epsilon)\right)\]
i.e.
\[\exists \epsilon,\, \forall \delta,\, \exists x\in B_{X}(x_{0},\delta) \quad f(x)\not\in B_{Y}(f(x_{0}), \epsilon)\]
Apply this with $\delta = 1, \frac{1}{2},\dots,\frac{1}{n},\dots$ to see that there exists $x_{1},x_{2},\dots,x_{n},\,$ such that
\[x_{n}\in B_{X}(x_{0}, 1 /n) \text{ and } f(x_{n})\not\in B_{Y}(f(x_{0}), \epsilon)\]

Then $x_{n}\to x_{0}$ in $X$ and $f(x_{n})\not\to f(x_{0})$ in $Y$, so, not 2
\end{proof}

\begin{thm}[Continuity and Open Sets]{def:continuity-open-sets}{}
    Let $(X, d_{X}),\,(Y, d_{Y})$ be metric spaces. A function $f: X \to Y $ is continuous iff the inverse image $f^{-1}(G)$ of any open subset $G$ of $Y$ is an open subset of $X$
\end{thm}

\begin{proof}
    Assume $f$ is continuous and let $G$ be an open subset of $Y$. Let $x_{0}\in f^{-1}(G)$. Then $f(x_{0})\in G$, therefore there exists a positive $\epsilon$ such that $B_{Y}(f(x_{0}), \epsilon)\subseteq G$. Since $f$ is continuous at $x_{0}$, there exists a positive $\delta$ such that, for all $x\in B_{X}(x_{0}, \delta)$ we have $f(x)\in B_{Y}(f(x_{0}), \epsilon)$, therefore $f(x)\in G$, therefore $x\in f^{-1}(G)$. This shows that $B_{X}(x_{0}, \delta)\subseteq f^{-1}(G)$.

    \noindent\rule{\textwidth}{0.2pt}
    Conversely, assume that the inverse image of every open subset of $Y$ is an open subset of $X$.

    Fix a point $x_{0}\in x$. We show that $f$ is continuous at $x_{0}$.

    Let $\epsilon$ be positive. The open ball $B_{Y}(f(x_{0}), \epsilon)$ is an open subset of $Y$, therefore $f^{-1}(B_{Y}(f(x_{0}), \epsilon))$ is an open subset of $X$ that contains $x_{0}$.

    Therefore, there exists a positive $\delta$ such that
    \[B_{X}(x_{0}, \delta)\subseteq f^{-1}(B_{Y}(f(x_{0}), \epsilon))\]
    For any $x\in B_{X}(x_{0}, \delta)$ we have $x\in f^{-1}(B_{Y}(f(x_{0}), \epsilon))$, therefore $f(x)\in B(f(x_{0}), \epsilon)$
\end{proof}

\textbf{Exercise}: Let $(X, d_{X}), (Y, d_{Y}), Z_{d_{Z}}$ be three metric spaces. Let $f : X \to Y$ and $g : Y\to Z$ be two continuous functions. Then $g\circ f : X \to Z$ is continuous

\newpage
\section{Topology!!!}
\subsection{Homeomorphisms and Topological Properties}
\begin{dfn}[Topological Space]{def:topological-space}{}
    A \textbf{topological space} is a set $X$ together with a family $\mathcal{T}$ of subsets of $X$ that has the following properties:
    \begin{itemize}
        \item $\emptyset,X\in \mathcal{T}$
        \item Any union of elements of $\mathcal{T}$ is an element of $\mathcal{T}$
        \item Any finite intersection of elements of $\mathcal{T}$ is an element of $\mathcal{T}$
    \end{itemize}
    $\mathcal{T}$ is called a \textbf{topology} and the elements of $\mathcal{T}$ are called \textbf{open sets}
\end{dfn}

\begin{dfn}[Continuity of Topological Spaces]{def:topological-continuity}{}
    Let $(X, \mathcal{T}_{X})$ and $(Y, \mathcal{T}_{Y})$ be two topological spaces. A function $f : X \to Y$ is said to be \textbf{continuous} iff for every $G$ in $\mathcal{T}_{Y}$ the pre-image $f^{-1}(G)$ is an element of $\mathcal{T}_{X}$.

    $f$ is said to be a \textbf{homeomorphism} iff it is a continuous bijection and its inverse is continuous.

    If such a homeomorphism exists then $(X, \mathcal{T}_{X})$ and $(Y, \mathcal{T}_{Y})$ are said to be \textbf{homeomorphic}
\end{dfn}

If $(X, \mathcal{T}_{X})$ and $(Y, T_{Y})$ are homeomorphic, and one of them is compact or connected or separable etc etc, then so is the other

Properties that are preserved by homeomorphisms are called topological properties

\subsection{Just kidding back to metric spaces}

\noindent\rule{\textwidth}{0.2pt}
\textbf{Example}: Let $(X, d_{X})$ be a discrete metric space and $(Y, d_{Y})$ be any metric space. Show that every function $f : X \to Y$ is continuous.

Indeed, the inverse image $f^{-1}(G)$ of any open subset $G$ of $Y$ is an open subset of $X$ (all subsets of $X$ are open)

\noindent\rule{\textwidth}{0.2pt}

\textbf{Example: }Let $X = \mathbb{R}$ equipped with the standard metric $d$, and $Y = \mathbb{R}$ equipped with the discrete metric $\rho$. Show the function $f : X \to Y,\, f(x) = x$ is not continuous.

\begin{proof}
    The set $\{0\}$ is open in $Y$, but the set $f^{-1}(\{0\}) = \{0\}$ is not open in $X$

    Actually, for any point $x_{0}\in X$, we have $x_{0} + \frac{1}{n}\to x_{0}\in X$, but
    \[f\left(x_{0} + \frac{1}{n}\right) = x_{0} + \frac{1}{n} \not\to x_{0} = f(x_{0}) \text{ in } Y\]
    Therefore, $f$ is not continuous at $x_{0}$
\end{proof}

Watch lecture recording 06/02 for examples of continuous functions

% TODO: actually write these things down - 06/03 30m


\begin{thm}[$d : X \times X \to \mathbb{R}$ is continuous]{thm:metric-space-continuity}{}
    Let $(X, d)$ be a metric space. The function $f: X \times X \to \mathbb{R} $ is continuous.

    $\mathbb{R}$ is equipped with the standard metric. $X \times X$ is equipped with the product metric
\end{thm}

\begin{proof}
    Fix $(x, x')\in X \times X$. We'll show that $d$ is continuous at $(x, x')$.

    Let $(x_{n}, x_{n}')\to (x, x')$ in $(X \times X, D)$. We'll show that
    \[d(x_{n}, x_{n}')\to d(x,x') \text{ in } \mathbb{R}\]
    % TODO: exercise 25?
    By exercise 25, $x_{n}\to x$ and $x_{n}' \to x'$ in $(X, d)$. By exercise 26, 
    \[\lvert d(x_{n}, x_{n}') - d(x,x') \rvert \le d(x_{n},x) + d(x_{n}', x')\to 0 + 0 = 0\]
\end{proof}


\noindent\rule{\textwidth}{0.2pt}

Let $X = Y = \mathbb{R}^{n}$, both equipped with the Euclidean metric $d_{2}$.

Let $A$ be an $n \times n$ matrix, and define $T: \mathbb{R}^{n} \to \mathbb{R}^{n} $ by $T(x) = Ax$. Then $T$ is continuous.

\begin{proof}
    Fix $x_{0}\in \mathbb{R}^{n}$. For all $x\in \mathbb{R}^{n}$ we have
    \begin{align*}
        d_{2}(T(x), T(x_{0})) &= \lVert T(x) - T(x_{0}) \rVert_{2} = \lVert T(x - x_{0}) \rVert_{2}\\
                              &= \lVert A(x - x_{0}) \rVert_{2} \le C\lVert x - x_{0} \rVert_{2} = Cd_{2}(x,x_{0})
    \end{align*}
    Where $C$ is a positive constant (independent of $x,x_{0}$).

    Let $\epsilon > 0$. Define $\delta = \frac{\epsilon}{C}$. for all $x$ with $d_{2}(x,x_{0}) < \delta$ we have
    \[d_{2}(T(x), T(x_{0})) \le Cd_{2}(x,x_{0}) < C\delta = \epsilon\]


\end{proof}

We need: For every $n \times n$ matrix $A$ there exists a constant $C$ such that, for all vectors $x\in \mathbb{R}^{n}$
\[\lVert Ax \rVert_{2} \le C\lVert x \rVert_{2}\]

\begin{proof}
    The $i$-th component of $Ax$ is $(Ax)_{i} = \sum_{j = 1}^{n} a_{ij}x_{j}$. By Cauchy-Schwarz,
    \[\lvert (Ax)_{i} \rvert^{2} \le \left(\sum_{j = 1}^{n} \lvert a_{ij} \rvert^{2}\right) \left(\sum_{j = 1}^{n} \lvert x_{j} \rvert^{2}\right) = \left(\sum_{j = 1}^{n} \lvert a_{ij} \rvert^{2}\right) \lVert  x \rVert^{2}_{2}\]
    Summing over $i$ we have
    \[\lVert Ax \rVert^{2}_{2} = \sum_{i = 1}^{n}\lvert (Ax)_{i} \rvert^{2} \le \underbrace{\left(\sum_{i = 1}^{n}\sum_{j = 1}^{n} \lvert a_{ij} \rvert^{2}\right)}_{= C^{2}} \lVert x \rVert^{2}_{2}\]
\end{proof}

\subsubsection{Continuity of linear operators between normed vector spaces}
Let $(X, \lVert \cdot \rVert_{X}),\,(Y, \lVert \cdot \rVert_{Y})$ be normed vector spaces. Recall that $d_{X} : X \times X \to \mathbb{R},\,d(x, x') = \lVert x - x' \rVert_{X}$, and $d_{Y} : Y \times Y \to \mathbb{R},\, d_{Y}(y, y') = \lVert  y - y' \rVert_{Y}$ are metrics

\begin{dfn}[Bounded Linear Operators]{def:bounded-linear-operators}{}
    A linear operator $T : X \to Y$ is said to be \textbf{bounded} iff there exists a positive constant $C$ such that, for all $x\in X$,
    \[\lVert T(x) \rVert_{Y} \le C \lVert x \rVert_{X}\]
\end{dfn}

\begin{thm}[Linear Operator Equivalence]{def:linear-operator-equiv}{}
    Let $T: X \to Y $ be a linear operator. The following are equivalent:
    \begin{enumerate}
        \item $T$ is continuous
        \item $T$ is continuous at $0$
        \item $T$ is bounded
    \end{enumerate}
\end{thm}

\begin{proof}
    $1 \implies 2$: Trivial

    $2 \implies 3$: Assume that $T$ is continuous at $0$. We wish to show:
    \[\exists C \forall x \lVert  T(x) \rVert_{Y} \le C \lVert x \rVert_{X}\]
    If not, then
    \[\forall C, \exists x \lVert  T(x) \rVert_{Y} > C \lVert x \rVert_{X}\]
    Observe that the $x$ is $\ne 0$. Apply with $C = 1,2,\dots,$ to see that there exists a sequence $(x_{n})_{n\in \mathbb{N}}\in X$ such that, for all $n$,
    \[\lVert T(x_{n}) \rVert_{Y} > n \lVert  x_{n} \rVert_{X}\]
    Define $x_{n}' = \frac{1}{n} \frac{x_{n}}{\lVert x_{n} \rVert_{X}}$. Then $d_{X}(x_{n}', 0) = \lVert x_{n}' \rVert_{X} = \frac{1}{n}\to 0$, therefore, $x_{n}' \to 0\in X$, but $T(x_{n}')\not\to 0 \in Y$ because $T(x_{n}')$ is bigger than $1$

    \noindent\rule{\textwidth}{0.2pt}
    $3 \implies 1$: Assume $T$ is bounded. Fix $x_{0}\in X$. Let $\epsilon > 0$. Define $\delta = \frac{\epsilon}{C}$. For all $x$ with $d_{X}(x,x_{0}) < \delta$ we have
    \begin{align*}
        d_{Y}(T(x), T(x_{0})) &= \lVert T(x) - T(x_{0}) \rVert_{Y}\\
                              &= \lVert T(x - x_{0}) \rVert_{Y}\\
                              &\le C \lVert x - x_{0} \rVert_{X}\\
                              &= C d_{X}(x,x_{0})\\
                              &< C\delta\\
                              &= \epsilon
    \end{align*}
\end{proof}

\noindent\rule{\textwidth}{0.2pt}
Watch lecture recording 09/02 for proofs on examples:
\begin{itemize}
    \item Let $(X, \lVert  \cdot \rVert)$ be a normed vector space and define $f: \mathbb{R} \times X \to X $ by $f(\lambda, x) = \lambda x$. Define $g: X \times X \to X $ by $g(x,y) = x + y$. $f$ and $g$ are continuous
\end{itemize}


\subsection{Fixed Points and Lipschitz}

\begin{dfn}[Lipschitz Functions]{def:lipschitz-functions}{}
    Let $(X, d_{X}),\,(Y, d_{Y})$ be metric spaces. A function $f : X \to Y$ is said to be a \textbf{Lipschitz} function iff there exists a constant $L$ such that for all $x,x'\in X$,
    \[d_{Y}(f(x), f(x')) \le L d_{X}(x,x')\]
    If $L < 1$, $f$ is said to be a \textbf{contraction}
\end{dfn}

\textbf{Note}: Magnus uses non-standard terminology here:
\begin{itemize}
    \item When the equation is satisifed and $L < 1$, Magnus calls $f$ a \textbf{strict contraction}
    \item He uses \textbf{contraction} for a functino $f$ that satisfies the weaker condition: for all $x, x' \in X$ with $x \ne x'$
        \[d_{Y}(f(x), f(x')) < d_{X}(x,x')\]
\end{itemize}

\begin{thm}[Lipschitz Continuity]{thm:lipschitz-continuity}{}
    Every Lipschitz function is continuous
\end{thm}

\begin{dfn}[Fixed Points]{def:fixed-points}{}
    A \textbf{fixed point} of a function $f: S \to S $ where $S$ is a non-empty set, is any element $x$ of $S$ such that $f(x) = x$

    Solving equations can sometimes be reduced to finding fixed points
\end{dfn}

Watch lecture recording 06/03 for more in-depth examples
\begin{itemize}
    \item Newton's Method for solving $f(x) = 0$
    \item Picard's Method for solving the Initial Value Problem
\end{itemize}

\begin{thm}[Metric Space Unique Fixed Points]{thm:complete-ms-fixed-point}{}
    Let $(X, d)$ be a complete metric space and let $f : X \to X$ be a contraction. Then $f$ has a unique fixed point
\end{thm}

\begin{proof}
    Let $x_{1}\in X$ and define $x_{n+1} = f(x_{n}),\, n = 1,2,\dots$

    $(x_{n})^{\infty}_{n = 1}$ is a Cauchy sequence. Observe first that, for all $n$,
    \[d(x_{n+1}, d+n) = d(f(x_{n}, f(x_{n-1})) \le L d(x_{n}, x_{n-1}))\]
    Therefore, for all $n$,
    \[d(x_{n+1}, x_{n}) \le Ld(x_{n}, x_{n-1})\le L^{2}d(x_{n-1},x_{n-2}) \le \cdots \le L^{n-1}d(x_{2}, x_{1})\]

    This goes on for like 10 more lines, watch 09/06 42 min
\end{proof}

\subsection{Equivalent Metrics}
\begin{dfn}[Equivalent Metrics]{def:equivalent-metrics}{}
    Two metrics on the same non-empty set $X$ are said to be \textbf{equivalent} iff they have teh same open sets
\end{dfn}

\textbf{Exercise:} Let $X$ be a non-empty set and $d_{1},\,d_{2}$ be two metrics on $X$. Prove that $d_{1}$ and $d_{2}$ are equivalent iff the identity function
\[i : (X, d_{1}) \to (X, d_{2})\]
is a homeomorphism (i.e. $i$ is continuous and its inverse $i^{-1} = i : (X, d_{2})\to (X, d_{1})$ is continuous)



\end{document}

% TODO: fix theorembox
% TODO: 
