\documentclass{article}
% \usepackage{showframe}

\usepackage[dvipsnames]{xcolor}
% custom colour definitions
\colorlet{colour1}{Red}
\colorlet{colour2}{Green}
\colorlet{colour3}{Cerulean}

\usepackage{geometry}
% margins
\geometry{
    a4paper, 
}

\usepackage{graphicx} % Required for inserting images
\usepackage{amsmath}
\usepackage{amsfonts}
\usepackage{amssymb}
\usepackage{preamble}
\usepackage{multicol}
\usepackage{lipsum}
\usepackage{float}
\usepackage[nodisplayskipstretch]{setspace}


% tikz and theorem boxes
\usepackage[framemethod=TikZ]{mdframed}
\usepackage{../thmboxes_col}
% \usepackage{thmboxes_col}


% Custom Definitions of operators
\DeclareMathOperator{\Ima}{im}
\DeclareMathOperator{\Fix}{Fix}
\DeclareMathOperator{\Orb}{Orb}
\DeclareMathOperator{\Stab}{Stab}
\DeclareMathOperator{\send}{send}
\DeclareMathOperator{\dom}{dom}

\title{Metric Spaces Notes}
\author{Leon Lee}


\begin{document}

\maketitle

\newpage
\section{Introduction to Metric Spaces}
\textbf{Metric} is another name for distance. A \textbf{Metric Space} is a set equipped with a metric.
A standard example is $\mathbb{R}$ with the standard metric
\[d(x,y) = \lvert x-y \rvert\]
We will now formally define what it means to have a metric

\begin{dfn}[Definition of a Metric]{metric-definition}{1.1}
    Let $X$ be a non-empty set. A function $d: X \times X \to \mathbb{R} $ is called a \textbf{metric} iff for all $x,\,y,\,z\in X$,
    \begin{itemize}
        \item $d(x,y)\ge 0$ and $d(x,y)=0 \iff x = y$\\
        \item $d(x,y)=d(y,x)$\\
        \item $d(x,y)\le d(x,z)+d(z,y)$ (Triangle Inequality)
    \end{itemize}
\end{dfn}
\end{document}
