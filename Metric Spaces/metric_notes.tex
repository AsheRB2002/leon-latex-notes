\documentclass{article}
% \usepackage{showframe}

% \usepackage[dvipsnames]{xcolor}
% custom colour definitions
% \colorlet{colour1}{Red}
% \colorlet{colour2}{Green}
% \colorlet{colour3}{Cerulean}

\usepackage{geometry}
% margins
\geometry{
    a4paper,
    bottom=70pt,
    % margin=70pt
}

\usepackage{graphicx} % Required for inserting images
\usepackage{amsmath}
\usepackage{amsfonts}
\usepackage{amssymb}
\usepackage{preamble}
\usepackage{multicol}
\usepackage{lipsum}
\usepackage{float}
\usepackage[nodisplayskipstretch]{setspace}

% tikz and theorem boxes
\usepackage[framemethod=TikZ]{mdframed}
\usepackage{../thmboxes_v2}
% \usepackage{thmboxes_col}


\usepackage{hyperref} % note: this is the final package
\parindent = 0pt

% Custom Definitions of operators
\DeclareMathOperator{\Ima}{im}
\DeclareMathOperator{\Fix}{Fix}
\DeclareMathOperator{\Orb}{Orb}
\DeclareMathOperator{\Stab}{Stab}
\DeclareMathOperator{\send}{send}
\DeclareMathOperator{\dom}{dom}

\title{Metric Spaces Notes}
\author{Leon Lee}
\renewcommand\labelitemi{\tiny$\bullet$}


\begin{document}

\maketitle
\newpage
\tableofcontents
\newpage

\section{Introduction to Metric Spaces}
\subsection{Defining a Metric}
\textbf{Metric} is another name for distance. A \textbf{Metric Space} is a set equipped with a metric.
A standard example is $\mathbb{R}$ with the standard metric
\[d(x,y) = \lvert x-y \rvert\]
We will now formally define what it means to have a metric

\begin{thm}[Definition of a Metric]{def:metric}{}
    Let $X$ be a non-empty set. A function $d: X \times X \to \mathbb{R} $ is called a \textbf{metric} iff for all $x,\,y,\,z\in X$,
    \begin{itemize}
        \item $d(x,y)\ge 0$ and $d(x,y)=0 \iff x = y$
        \item $d(x,y)=d(y,x)$
        \item $d(x,y)\le d(x,z)+d(z,y)$ (Triangle Inequality)
    \end{itemize}
    A non-empty set $X$ equipped with a metric $d$ is called a \textbf{metric space}
\end{thm}

\subsection{Examples of Metric Spaces}
We can construct a metric space using the \textbf{Absolute value} equipped with the standard triangle inequality
\begin{xmp}[The Real Line]{xmp:real-line}{}
    Let $X = \mathbb{R}$. Define our metric $x: X \times X \to \mathbb{R} $ by
    \[d(x,y)=\lvert x-y \rvert\]
    The first two properties are fairly trivial. The third property follows using the regular triangle inequality
    \[d(x,y) = \lvert x - y \rvert = \lvert (x - z) + (z - y) \rvert \,\le\, \lvert x - z \rvert + \lvert z - y \rvert = d(x,z) + d(z,y)\]
\end{xmp}

\textbf{Remark}: This can be extended not just in $\mathbb{R}^{2}$, but to all $\mathbb{R}^{n}$. By induction,
\[\lvert x_{1} + \cdots + x_{N} \rvert \le \lvert x_{1} \rvert + \cdots + \lvert x_{N} \rvert\]
If $\displaystyle\sum_{n = 1}^{\infty}x_{n}$ converges absolutely, let $N\to +\infty$ to see that
\[\left\lvert \sum_{n = 1}^{\infty}x_{n} \right\rvert \le \sum_{n = 1}^{\infty} \lvert x_{n} \rvert\]

\newpage
A second example is the \textbf{Euclidean Plane}. The metric is defined using the \textbf{inner product} and the \textbf{norm}.

\begin{dfn}[Inner Product]{def:inner-product}{}
    The \textbf{inner product} is defined as
    \[\langle x,y \rangle = x_{1}y_{1} + x_{2}y_{2}\]
    Properties of the inner product:
    For all vectors $x,\,y,\,z\in \mathbb{R}^{2}$ and all real scalars $a,\,b$,
    \begin{itemize}
        \item $\langle x,x \rangle\ge 0$ and $\langle x,x \rangle = 0 \iff x = 0$
        \item $\langle x,y \rangle$ = $\langle y,x \rangle$
        \item $\langle ax+by,z \rangle = a\langle x,z \rangle + b\langle y,z \rangle$
    \end{itemize}
\end{dfn}

\textbf{Remark}: This is basically a formalisation of the dot product

\begin{dfn}[Norm]{def:norm}{}
    The \textbf{norm} is defined as:
    \[\lVert x \rVert_{2} = \langle x,x \rangle^{1/2} = \sqrt{x^{2}_{1} + x^{2}_{2}}\]
    Properties of the norm: For all $x,\,y\in\mathbb{R}^{2}$, $a\in\mathbb{R}$
    \begin{itemize}
        \item $\lVert x \rVert_{2} \ge 0$ and $\lVert x \rVert_{2} = 0 \iff x = 0$
        \item $\lVert ax \rVert_{2} = \lvert a \rvert\lVert x \rVert_{2}$
        \item $\lVert x+y \rVert_{2} \le \lVert x \rVert_{2}+\lVert y \rVert_{2}$ (triangle inequality)
    \end{itemize}
\end{dfn}
\textbf{Remark}: This is a formalisation of the "length of a vector"

With these two properties, we can now define the \textbf{Euclidean Metric}
\begin{xmp}[Euclidean Metric]{xmp:euclidean}{}
    For all $x = (x_{1},x_{2}),\, y=(y_{1},y_{2})\in \mathbb{R}^{2}$, define
    \[d_{2}(x,y) = \lVert x-y \rVert_{2}=\sqrt{(x_{1}-y_{1})^{2} + (x_{2}-y_{2})^{2}}\]
\end{xmp}

\textbf{Remark}: Derivation of the triangle inequality is basically the same as Example \ref{xmp:real-line}. 
\[d_{2}(x,y) = \lVert x - y \rVert_{2} = \lVert (x-z) + (z-y) \rVert_{2} \,\le\, \lVert x-z \rVert_{2} + \lVert z-y \rVert_{2} = d_{2}(x,z) + d_{2}(z, y)\]

\subsubsection{Proof of the euclidean triangle inequality}
W.T.S: 
\[\lVert x+y \rVert_{2} \le \lVert x \rVert_{2} + \lVert y \rVert_{2}\]

\textbf{Proof}: Square both sides
\begin{align*}
    \text{LHS}^{2} &= \langle x + y, x + y \rangle  & \text{RHS}^{2}&= \lVert x \rVert^{2}_{2} + \lVert y \rVert^{2}_{2} + 2\lVert x \rVert_{2}\lVert y \rVert_{2}\\
                   &= \langle x,x \rangle + 2\langle x,y \rangle + \langle y,y \rangle \\
                   & = \lVert x \rVert^{2}_{2} + 2\langle x,y \rangle + \lVert y \rVert^{2}_{2}
\end{align*}
\newpage

Discarding the equal terms, we get
\begin{align*}
    \textcolor{red}{\lVert x \rVert^{2}_{2}} + 2\langle x,y \rangle + \textcolor{red}{\lVert y \rVert^{2}_{2}} &\le \textcolor{red}{\lVert x \rVert^{2}_{2}} + \textcolor{red}{\lVert y \rVert^{2}_{2}} + 2\lVert x \rVert_{2}\lVert y \rVert_{2}\\
    \langle x,y \rangle &\le \lVert x \rVert_{2}\lVert y \rVert_{2} \\
    \text{i.e. } x_{1}y_{1} + x_{2}y_{2} &\le \sqrt{x^{2}_{1} + x^{2}_{2}}\sqrt{y^{2}_{1}+y^{2}_{2}}
\end{align*}
This is the \textbf{Cauchy-Schwarz Inequality}. Various ways to prove this (watch lecture 1)

\begin{xmp}[Complex Plane]{xmp:complex-plane}{}
    Let $X = \mathbb{C},\, d: \mathbb{C} \times \mathbb{C} \to \mathbb{R} $
    \[d(z,w) = \lvert z - w \rvert\]
    If $z = a+ib, w = c+id,\, a,b,c,d\in\mathbb{R}$, then
    \[z - w = (a - c) + i(b - d)\]
    therefore,
    \[d(z,w) = \sqrt{(a-c)^{2} + (b-d)^{2}}\]
\end{xmp}

\begin{dfn}[$n$-dimensional Euclidean space]{def:ndim-euclidean-space}{}
    Let $X = \mathbb{R}^{n} = \{(x_{1},x_{2},\dots,x_{n}) : x_{1},x_{2},\dots,x_{n}\in\mathbb{R}\}$
    \newline
    For $x = (x_{1},x_{2},\dots,x_{n})$, $y=(y_{1},y_{2},\dots,y_{n})$ in $\mathbb{R}^{n}$, define
    \[\langle x,y \rangle = x_{1}y_{1} + x_{2}y_{2} + \cdots + x_{n}y_{n} \text{ (inner product)}\]
    \textbf{Properties of $n$-inner product}: For all vectors $x,y,z\in\mathbb{R}^{n}$ and all real scalars $a,b$,
    \begin{itemize}
        \item $\langle x,x \rangle\ge 0$ and $\langle x,x \rangle=0 \iff x = 0$
        \item $\langle x,y \rangle = \langle y,x \rangle$
        \item $\langle ax+by,z \rangle = a\langle x,z \rangle + b\langle y,z \rangle$
    \end{itemize}
    For $x = (x_{1},x_{2},\dots,.x_{n})\in\mathbb{R}^{n}$ define
    \[\lVert x \rVert_{2} = \langle x,x \rangle^{1/2} = \sqrt{x^{2}_{1} + x^{2}_{2} + \cdots + c^{2}_{n}}\text{(norm)}\]
    \textbf{Properties of $n$-norm}: For $x,y\in\mathbb{R}^{n},\,a\in\mathbb{R}$,
    \begin{itemize}
        \item $\lVert x \rVert_{2}\ge 0$ and $\lVert x \rVert_{2}=0 \iff x = 0$
        \item $\lVert ax \rVert_{2}=\lvert a \rvert\lVert x \rVert_{2}$
        \item $\lVert x+y \rVert_{2}\le \lVert x \rVert_{2} + \lVert y \rVert_{2}\text{ (triangle inequality)}$
    \end{itemize}
\end{dfn}

\begin{xmp}[Metric in $n$-dim euclidean space]{xmp:ndim-euclidean-space}{}
    For $x = (x_{1},x_{2},\dots,x_{n}),\,y=(y_{1},y_{2},\dots,y_{n})$ in $\mathbb{R}^{n}$, define
    \begin{align*}
        d_{2}(x,y) &= \lVert x - y \rVert_{2} \\
                   &= \sqrt{(x_{1}-y_{1})^{2} + (x_{2}-y_{2})^{2} + \cdots + (x_{n} - y_{n})^{2}}
    \end{align*}
    Triangle inequality, cauchy schwarz, yadda yadda same as $2$-dim case
\end{xmp}



\end{document}
