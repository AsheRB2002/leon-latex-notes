\documentclass{article}
% \usepackage{showframe}

% \usepackage[dvipsnames]{xcolor}
% custom colour definitions
% \colorlet{colour1}{Red}
% \colorlet{colour2}{Green}
% \colorlet{colour3}{Cerulean}

\usepackage{geometry}
% margins
\geometry{
    a4paper,
    bottom=70pt,
    % margin=70pt
}

\usepackage{graphicx} % Required for inserting images
\usepackage{amsmath}
\usepackage{amsfonts}
\usepackage{amssymb}
\usepackage{preamble}
\usepackage{multicol}
\usepackage{lipsum}
\usepackage{float}
\usepackage[nodisplayskipstretch]{setspace}

% tikz and theorem boxes
\usepackage[framemethod=TikZ]{mdframed}
\usepackage{../thmboxes_v2}
% \usepackage{thmboxes_col}


\usepackage{hyperref} % note: this is the final package
\parindent = 0pt

% Custom Definitions of operators
\DeclareMathOperator{\Ima}{im}
\DeclareMathOperator{\Fix}{Fix}
\DeclareMathOperator{\Orb}{Orb}
\DeclareMathOperator{\Stab}{Stab}
\DeclareMathOperator{\send}{send}
\DeclareMathOperator{\dom}{dom}

\title{Dynamics and Vector Calculus Notes}
\author{Leon Lee}
\renewcommand\labelitemi{\tiny$\bullet$}


\begin{document}

\maketitle
\newpage
\tableofcontents
\newpage

\section{Couple Oscillations and normal modus}

% \begin{figure}[h!]
%     \centering
%     \includegraphics[width=\linewidth]{}
%     \caption{idk how to do diagrams in LaTeX}
%     \label{spring}
% \end{figure}
some diagram idk

\noindent\rule{\textwidth}{0.2pt}

where $x_{1}$ and $x_{2}$ are displacements from equilibrium

\textbf{For mass 1}
\begin{itemize}
    \item Force to the left: $-k_{1}x_{1}$
    \item Force to the right: $-k_{2}(x_{2}-x_{1})$
\end{itemize}

\[m \frac{d^{2}x_{1}}{dt^{2}} = -k_{1}x_{1} + k_{2}(x_{2} - x_{1}) - k_{3}x_{2}\]

Write this in matrix form
\[m \frac{d^{2}}{dt^{2}}\begin{pmatrix}
    x_{1} \\
    x_{2}
\end{pmatrix} = \begin{pmatrix}
k_{1} + k_{2} & -k_{2} \\
-k_{2} & k_{2} + k_{3}
\end{pmatrix} \begin{pmatrix}
    x_{1} \\
    x_{2}
\end{pmatrix} \implies m \frac{d^{2}x}{dt^{2}} = -K x\]

\begin{dfn}[Normal Mode Solution]{normal-mode}{}
    \textbf{Normal Mode Solution}: All co-ordinates (here $x_{1},\,x_{2}$) oscillate with the same frequency
\end{dfn}

$x(t) = \cos{(\omega t - \phi)} \underline{b}$

$\underline{b}$ is constant vector, $\omega$ to be determined

sub in matrixeq??
\begin{align*}
    &-m \omega^{2} \cos(\omega t - \phi) \underline{b} + K \cos(\omega t - \varnothing) \underline{b} = 0\\
    & -m \omega^{2} \underline{b} + K \underline{b} = 0 \to K \underline{b} = \lambda \underline{b} \quad \lambda = m \omega^{2}
\end{align*}
where $\lambda$ is eigenvalue, and $b$ is eigenvector

For simplicity, take $k_{1} = k_{2} = k_{3} = k$

Then,
\[K = \begin{pmatrix}
    2k & -k \\
    -k & 2k
\end{pmatrix} \qquad (K - \lambda \mathbb{1}) \underline{b} = 0 \implies \lvert k - \lambda \mathbb{1} \rvert = 0\]

\[\begin{vmatrix}
    2k - \lambda & -k \\
    -k & 2k - x
\end{vmatrix} = 0 \implies (2k - \lambda)^{2} - k^{2} = 0\]

This is called the "Characteristic Equation"

$(2k - \lambda) = \pm k \quad \lambda = 2k \mp k$

Therefore, $\lambda = k, 3k$

Mode A: $\lambda_{A} = k \quad (K - k \mathbb{1}) \underline{b} = 0$
\[\begin{pmatrix}
    k & -k \\
    -k & k
\end{pmatrix} \begin{pmatrix}
    b_{1} \\
    b_{2}
\end{pmatrix} = 0 \quad \underline{b}_{A} = Ct \begin{pmatrix}
    1 \\
    1
\end{pmatrix}\]

Usually, choose a constant s.t. $\underline{b} \cdot \underline{b} = 1$

Mode B: $\lambda_{A} = 3k \quad (K - 3k \mathbb{1}) \underline{b} = \begin{pmatrix}
    -k & -k \\
    -k & -k
\end{pmatrix}\begin{pmatrix}
    b_{1} \\
    b_{2}
\end{pmatrix} = 0$

and some stuff more i forgor to write

\noindent\rule{\textwidth}{0.2pt}

[diagram thing]

Normal mode $\underline{x}(t) = \underline{b}\cos(\omega t - \phi)\to (K - \kappa \mathbb{1})\underline{b} = 0 \quad \lambda = m \omega^{2}$
\[\lambda_{A} = k,\, \underline{b}_{A} = \frac{1}{\sqrt{2}}\begin{pmatrix}
    1\\
    1
\end{pmatrix} \quad \lambda_{B} = 3k,\, \underline{b}_{B} = \frac{1}{\sqrt{2}}\begin{pmatrix}
    1\\
    -1
\end{pmatrix}\]

So we have $2$ independent solutions 

General solution: $\underline{x}(t) = A \underline{b}_{A} \cos(\omega_{A} t - \phi_{A}) + B \underline{b}_{B}\cos(\omega_{B}t - \varnothing_{B})$

So there are $4$ constants $A,\,B,\,\phi_{A},\,\varnothing_{B}$ to be fixed


\subsection{Motion in Normal modes}

\begin{align*}
    \text{Mode A} & x_{1} = x_{2} \quad \text{"unphase"} \quad \omega_{A} = \left(\frac{k}{m}\right)^{2} \\
    \text{Mode B} & x_{1} = -x_{2} \quad \text{"antiphase"} \quad \omega_{A} > \omega{B}
\end{align*}

\textbf{Normal Co-ordinates}
Take scalar product

\begin{align*}
    (1, 1) \begin{pmatrix}
    x_{1} \\
    x_{2}
\end{pmatrix} &= x_{1} + x_{2} = 2A\cos(\omega_{A} - \phi_{A}) \\
    (1, -1) \begin{pmatrix}
    x_{1} \\
    -x_{2}
\end{pmatrix} &= x_{1} + x_{2} = 2B\cos(\omega_{B}t - \phi_{B}) \\
\end{align*}

\noindent\rule{\textwidth}{0.2pt}

\textbf{Define}

\begin{align*}
    z_{1} = \frac{1}{\sqrt{2}}(x_{1} + x_{2}) = \alpha^{1}\cos(w_{A}t - \phi) \quad z_{1}+\omega^{2}_{A} z_{1} = 0 \quad\text{(SHO)} \\
    z_{2} = \frac{1}{\sqrt{2}}(x_{1} - x_{2}) = \beta^{1}\cos(w_{B}t - \phi) \quad z_{2}+\omega^{2}_{B} z_{2} = 0\quad\text{(SHO)}
\end{align*}

$z_{1}$ and $z_{2}$ are each independent simple harmonic motions, and energy is preserved in each one

\[E_{A} = \frac{1}{2} m (z_{1})^{2} + \frac{1}{2} k z^{2}_{1} = \text{constant in time}\]

\subsection{Summmary: properties of Normal Modes}
\[\]
\begin{itemize}
    \item $\underline{x}_{\alpha} = A_{\alpha} \underline{b}_{A} \cos(\omega_{\alpha} t - \phi_{\alpha}))$
    \item All coordinates oscillate at the same frequency
    \item constants $A_{\alpha}, \phi_{\alpha}$ are fixed by  ic (???)
    \item General motion is superposition of normal modes
    \item Normal coordinates $z_{\alpha} = \underline{b}_{\alpha} \cdot \underline{x}$
    \item Transforming to the noraml coordinates $\to$ diagonalise $k$ (see notes i.e. ask alice or fiona for them)
    \item Energy in each normal mode conserved, mode with lowest $\omega$ is the most symmetric
\end{itemize}

\newpage
\subsection{Coupled Pendulum}
[a diagram]

pendulum thing
\[ml \frac{d^{2}\theta}{dt^{2}} = -ml \omega^{2}_{?} \theta\quad \omega_{\theta} = \left(\frac{g}{l}\right)^{1/2}\]

Add in the force from the spring exension:
\[x_{2} - x_{1} = l(\sin\theta_{2} - \sin\theta_{3}) \approx l(\theta_{2} - \theta)\]

\begin{align*}
    m \frac{d^{2}\theta_{1}}{dt^{2}} &= -m\omega_{0}^{2}\theta_{1} + k(\theta_{2} - \theta_{1}) \\
    m \frac{d^{2}\theta_{2}}{dt^{2}} &= -m\omega_{0}^{2}\theta_{2} + k(\theta_{2} - \theta_{1})
\end{align*}

Putting it in vector form thing
\[m \frac{d^{2}}{dt^{2}}\begin{pmatrix}
    \theta_{1} \\
    \theta_{2}
\end{pmatrix} = - \begin{pmatrix}
m\omega^{2}_{0} + k & -k \\
-k & m\omega_{0}^{2} + k
\end{pmatrix} \begin{pmatrix}
    \theta_{1} \\
    \theta_{2}
\end{pmatrix}\]

Normal mode $\underline{\theta} = \underline{b}\cos(\omega t) \quad -m\omega_{2} \underline{b} + K\underline{b} = 0$

eigenvalue problem $K\underline{b} = \lambda \underline{b} \quad \lambda = m\omega^{2}$

\[\det(K - \lambda \mathbb{1}) = 0 \quad \begin{vmatrix}
    m\omega_{0}^{2} + k - \lambda & -k \\
    -k & m\omega_{0}^{2} - k - \lambda
\end{vmatrix} = 0\]

\[(mw\omega^{2}_{0} + k - \lambda)^{2} - k^{2} = 0 \quad \lambda_{\Delta} = m\omega_{0}^{2} \quad \lambda_{B} = m\omega^{2}_{0} + 2k\]

Eigenvctors
\begin{align*}
    \lambda_{A} &= m\omega^{2}_{0} = \begin{pmatrix}
    k & -k\\
    -k & k
\end{pmatrix} \underline{b}_{A} = 0 \quad \underline{b}_{A} = \frac{1}{\sqrt{2}}\begin{pmatrix}
    1\\
    1
\end{pmatrix} \quad \text{ inphase} \quad \omega = \omega_{0} \\
        \lambda_{B} &= m\omega^{2}_{0} = \begin{pmatrix}
    k & -k\\
    -k & -k
\end{pmatrix} \underline{b}_{B} = 0 \quad \underline{b}_{B} = \frac{1}{\sqrt{2}}\begin{pmatrix}
    1\\
    -1
\end{pmatrix} \quad \text{ antiphase} \quad \omega_{B}^{2} = w_{0}^{2} + \frac{2k}{m}
\end{align*}

\subsubsection{Mass Matrix}
\begin{align*}
    m_{1}x_{1} &= (k_{1} + k_{2}) x_{1} + k_{2}x_{2}\\
    m_{2}x_{2} &= k_{2}x_{1} - (k_{3} + k_{2})x_{2}
\end{align*}

NOTE: have defo missed some double dot $x$ at some points

Write this as $M\underline{\ddot{x}} = -K\underline{x} \quad M = \begin{pmatrix}
    m_{1} & 0 \\
    0 & m_{2}
\end{pmatrix}$

Normal mode $\underline{x}(t) = \underline{b}\cos(\omega t - \pi) \quad -\omega^{2} M \underline{b} = -K \underline{b} \quad (K - \lambda M)\underline{ b} = 0$

$(K - \lambda M) \underline{b} = 0$ for nontrivial sol $^{n}$(??) $\det(K - \lambda M) = 0$

\[ \lambda = \omega^{2} \quad \begin{vmatrix}
    k_{1} + k_{2} - \lambda m_{1} & -k_{2} \\
    -k_{2} & k_{1} k_{2} - \lambda m_{2}
\end{vmatrix} = 0\]

$\implies$ quadratic for $\lambda$. For equal $k$,
\[(2k - \lambda m_{1})(2k - \lambda m_{2}) - k^{2} = 0 \quad \text{(quadratic)}\]

\newpage

\subsection{Line integral :)}

idk what's happening but line integral

\[\Gamma(x) = \int_{a}^{a + \frac{\pi}{2}} \sin^{2}\lambda d\lambda = \int_{a}^{a + \frac{\pi}{2}} \cos^{2}\lambda d\lambda = \frac{1}{2} \int_{a}^{a + \frac{\pi}{2}} \sin^{2}\lambda + \cos^{2}\lambda d\lambda = \frac{\pi}{4}\]
\[ = \frac{\pi}{4} - \frac{1}{3} \sin^{3}\lambda\cos\lambda \bigg\rvert^{\frac{\pi}{2}}_{0} - \frac{1}{3} \underbrace{\int_{0}^\frac{{\pi}{2}} \sin{4}\lambda d\lambda}_{I} = \frac{\pi}{4} - \frac{I}{3} = I\]

random facts
\begin{itemize}
    \item $\underline{\Delta} \times \underline{\Delta} \phi = 0$
    \item $\underline{\Delta} \cdot (\underline{ \Delta \times \underline{a}}) = 0$
\end{itemize}

\subsection{Surface Integrals}

(shoutout to the generalised stoke's theorem, he got me fr fr)

\begin{dfn}[Paramatric form of the surface integral]{def:surface-int-param}{}
    \begin{align*}
        \underline{\Lambda} &= \underline{\Lambda}(u,v)\\
                            &= x_{1}(u,v)\underline{e}_{1} + x_{2}(u,v)\underline{e}_{2} + x_{3}(u,v)\underline{e}_{3}
    \end{align*}
\end{dfn}

\textbf{Example:} Sphere (in spherical coordinates) idk how to draw diagrams

\begin{align*}
    x_{1}(\theta, \phi) &= \sin\theta\cos\phi\\
    x_{2}(\theta, \phi) &= \sin\theta\sin\phi\\
    x_{3}(\theta, \phi) &= \cos\theta
\end{align*}

\[d\underline{r} = \underbrace{\partial_{X}\underline{r} du}_{d\underline{r}_{u}} + \underbrace{\partial_{X}\underline{r} dv}_{d\underline{r}_{v}}\]

\[d\underline{S} = d\underline{R}_{u} \times d\underline{r}_{v} = \text{"area of infinitessimal parallelogram"}\]

Actual line integral equation
\begin{dfn}[Line Integral Equation]{def:line-int}{}
    \[\int_{S} \underline{a} \cdot d \underline{S} = \iint \underline{a} \cdot (\partial_{u}\underline{r} \times \partial_{v}\underline{r})du dv\]
\end{dfn}

\textbf{Remarks}z
\begin{itemize}
    \item $\underline{\vec{n}} \propto \partial{u}\underline{r} \times \partial_{v}\underline{r}$
    \item ambiguity in orientation ($u \leftrightarrow v$ interchase)
        \begin{itemize}
            \item (circle): closed surface, choose $\underline{\vec{u}}$ outwards
            \item open surface: can do either - choose one. In Stoke's theorem, there will be a double ambiguity
                \begin{itemize}
                    \item Orientation of $S$
                    \item Direction of line integral
                \end{itemize}
        \end{itemize}
    \item For applications, $\partial_{u}\underline{r} \delta_{v}\underline{r}$. 
        e.g. spherical coords: $\partial_{\theta}\underline{r} \times \partial_{\phi}\underline{r} \propto \partial_{r}\underline{r}$
\end{itemize}


\begin{dfn}[somethinglinear coordinates $I$, flux \& surface]{def:idk}{}
    \[(x_{1},\, x_{2},\,x_{3}) \to (x,y,z)\]
\end{dfn}

\begin{dfn}[Plan polar coordinates]{def:polar-coords}{}
    \begin{align*}
        x_{1}(\rho, \phi) &= \rho\cos\phi & \quad & \rho = \sqrt{x_{1}^{2} + y_{2}^{2}}\in [0, \infty) \\
        x_{2}(\rho, \phi) &= \rho\sin\phi & \quad & tg\phi = x_\frac{{2}}{x_{1}}\in [0, \infty)
    \end{align*}

    \[\underline{\Gamma}(\rho, \phi) = x_{1}(\rho, \phi)\underline{e}_{1} + \underline{x}_{2}(\rho,\phi)\underline{e_{2}}\]
    
\end{dfn}

\[\underline{e}_{\rho} = \frac{\partial_{\rho}\underline{r}}{\lvert \partial_{\rho}\underline{r} \rvert} = \cos\phi\underline{e_{1}} + \sin\phi \underline{e_{2}} = \frac{1}{\rho}\underline{r} (\text{special case final line})\]

\textbf{Remarks}

\begin{itemize}
    \item $\underline{r} \ne \rho \underline{e}_{\rho} + \phi\underline{e}_{\rho}$
    \item generally, $\underline{a} = \underline{a}_{\rho}\underline{\epsilon}_{\rho} + a_{\phi}\underline{e}_{\phi},\, \alpha_{\rho, \phi} = \underline{a} \cdot \underline{e}_{\rho, \phi}$
    \item new aspect: $\{ \underline{e}_{\rho}, \underline{e}_{\rho}\}$ is position dependent
\end{itemize}


\begin{dfn}[Cylindrical Coordinates]{def:cylindrical-coords}{}
    \begin{align*}
        x_{1} &= \rho\cos\phi\\
        x_{2} &= \rho\sin\phi\\
        x_{3} &= z
    \end{align*}

    \begin{align*}
        \underline{\Gamma}(\rho,\phi,z) &= \rho\cos\phi \underline{e}_{1} + \rho\sin\phi\underline{e}_{1} + z\underline{e}_{3}\\
                                        &= \rho\underline{e}_{\rho} + z\underline{e}_{z} \text{ (special case)}
    \end{align*}

    \textbf{Note:} $\{\underline{e}_{\rho}, \underline{e}_{\phi}, \underline{e}_{\zeta}\}$ forms a right-hand orthogonal basis
\end{dfn}

\newpage
\section{filler}
\section{more filler}
\subsection{filler}
\subsection{The Vector Potential}

Results ahead...

\[\underline{\nabla} \times \underline{a} = 0 \iff \exists\phi \text{ s.t. } \underline{a} = \underline{\nabla} \phi ;\, \phi = \int_{0}^{1} d\lambda (\underline{a} ( \lambda \underline{r}) \cdot \underline{r})\]
\[\underline{\nabla} \times \underline{B} = 0 \iff \exists \underline{A} \text{ s.t. } \underline{B} = \underline{\nabla} \underline{A} ;\, \underline{A} = \int_{0}^{1} d\lambda (\underline{B} ( \lambda \underline{r}) \cdot \underline{r}\lambda)\]

\begin{thm}[Helmholtz Theorem]{thm:helmholtz}{}
    Smooth $\underline{Q}$ decomposes (not unique)

    \[\underline{Q} = \underline{\nabla} g + \underline{\nabla} \times \underline{G}\]

    Conservative curl-free div-free vector free
\end{thm}

\textbf{Note:} related - Hodge decomposition (valid more generally) whatever that is

\noindent\rule{\textwidth}{0.2pt}

\textbf{Example}: 


\begin{align*}
    \underline{B} &= \underline{c} \times \underline{r},\quad \underline{\nabla} \cdot \underline{B} = \partial_{x_{i}} \epsilon_{ijk} c_{j} \times k\\
                  &= \delta_{k}\epsilon_{ijk}c_{j} = 0
\end{align*}

note: what?

\begin{align*}
    \underline{A} = -\underline{r} \times \int_{0}^{1} \underline{B}(\lambda \underline{r}) d\lambda &= -\underline{r} \lambda (\underline{c} \times \underline{r}) \underbrace{\int_{0}^{1} d\lambda \lambda^{2}}_{1 /3}\\
    &= \frac{1}{3} ((\underline{r} \cdot \underline{c})\underline{r} - r^{2}\underline{c})
\end{align*}

\[\underline{a} \times (\underline{b} \cdot \underline{ c})\]

can't keep up with the guy lol

\newpage

\textbf{Formal Check of 2} (second formula, TODO: acstually figure out how to mark number lol)


\[\underline{\nabla} \times \underline{\Delta} = \underline{\nabla} \times \int_{0}^{1} \underline{B} (\lambda \underline{r}) \times \underline{r} \lambda d\lambda = \int_{0}^{1} f(\lambda, \underline{r}) \lambda d\lambda\]

\begin{align*}
    f(\lambda, \underline{r}) &=^\text{(F)} [\underline{\nabla} \cdot \underline{r} \quad \underline{B}(\lambda \underline{r}) + \underline{r} \cdot \underline{\nabla} \quad \underline{B}(\lambda \underline{r})] - [\underbrace{(\underline{\nabla} \cdot \underline{B})}_{0} \underline{r} + \underbrace{(\underline{B} (\lambda \underline{r}) \cdot \underline{\nabla}) \underline{r}}_{\underline{B}}]\\
                              &= 2\underline{B}(\lambda \underline{r}) + \underbrace{x_{i}\partial_{x_{i}} \quad \underline{B}(\lambda x_{1}, \lambda x_{2}, \lambda x_{3})_{\lambda\partial\lambda \underline{B}(\lambda x_{1}, \lambda x_{2}, \lambda x_{3})}}\\
                              &= \int_{0}^{1} [\underbrace{2\underline{B}(\lambda\underline{r})\lambda}_{I_{n}} + \underbrace{\lambda^{2}\partial_{\lambda} \underline{B}(\lambda\underline{r})}_{I_{2}}] d\lambda \\
I_{2} &= \underbrace{\lambda^{2}\underline{B}(\lambda\underline{r})}_{\underline{B}(\underline{r})} \big\rvert^{1}_{0} - \underbrace{\int_{0}^{1} 2\lambda \underline{B}(\lambda\underline{r}) d\lambda}_{\underline{I}_{n}} = \underline{B} - \underline{I_{n}} \\
\underline{\nabla} \times \underline{A} &= \underline{I_{n}} + \underline{I_{2}} = \underline{I_{n}} + (\underline{B} - \underline{I_{n}}) = \underline{B}
\end{align*}

dunno what that equation means :P


\subsection{Orthogonal Curvilinear Co-ordinates (OCC)}

e.g. spherical co-ords

\begin{align*}
    x &= r \sin\theta\cos\phi\\
    y &= r\sin\theta\sin\pi\\
    z &= r\cos\theta
\end{align*}

In general: 

$u_{i} = u_{i}(x_{1}, x_{2}, x_{3})$, $i = 1\dots 3$ (3 single valued invertible functions of 3 variables)

$x_{i} = x_{i}(u_{1}, u_{2}, u_{3})$

\begin{align*}
    r &= \sqrt{x^{2} + y^{2} + z^{2}} & r = \text{ const $\implies$ Sphere}\\
    \theta &= \cos ^{-1} (z /r) & \theta = \text{ const $\implies$ Cones}\\
    \phi &= tg^{-1}(y \ x) & \phi = \text{ const $\implies$ Planes}
\end{align*}

\subsubsection{OCC}

\begin{itemize}
    \item $\partial u_{i} \underline{r} \cdot \partial_{u_{j}} \underline{r} = 0 \quad i\ne j \quad \text{ orthogonality}$
    \item $\text{Scale Factor: }h_{i} = \lvert \partial_{u_{i}} \underline{r}\rvert\quad \text{norm}$
    \item $\underline{e}_{u_{i}} = \frac{1}{h_{i}}\partial_{u_{i}}\underline{r} \qquad \underline{e}_{u_{i}} \cdot \underline{e}_{u_{j}} = \delta_{ij}$
\end{itemize}

\textbf{Examples}:
\begin{enumerate}
    \item Cartesian Coordinates: $\underline{r} = x_{i}\underline{e}_{i} \quad h_{i} = \lvert \partial_{x_{i}} \underline{r} \rvert = 1$
    \item Spherical: $\underline{r} = r [\sin\theta\cos\phi\underline{e}_{1} + \sin\theta\sin\phi\underline{e}_{2} + \cos\theta\underline{e}_{3}]$

    \begin{align*}
    \partial_{r}\underline{r} &= [] \implies h_{r} = \lvert \partial_{r}\underline{r} \rvert = 1k \\
    \partial_{\theta}\underline{r} &= r\sin\theta(\underbrace{-\sin\phi\underline{c}_{1} + \cos\phi\underline{e}_{2})}_{\underline{e}_{f}} \\
    h_{\phi} &= \lvert \partial_{\phi}\underline{r} \rvert = r\sin\theta
    \end{align*}

\item Cylindrical Coordinates: $\underline{r} = \rho\cos\phi\underline{e}_{1} + \rho\sin\phi\underline{e}_{2} + z\underline{e}_{3}$
    \begin{align*}
        \partial_{\rho}\underline{r} &= \underbrace{\cos\phi\underline{e}_{1} + \sin\phi\underline{e}_{2}}_{\underline{e}_{\rho}} & h_{\rho} = 1 \\
        \partial_{\phi}\underline{r} &= \rho (\underbrace{-\sin\phi\underline{e}_{1} + \cos\phi\underline{e}_{2}}_{\underline{e}_{\phi}}) & h_{\phi} = \rho \\
        \partial_{z}\underline{r} &= \underline{e}_{3} = \underline{e}_{z} & h_{z} = 1
    \end{align*}
\end{enumerate}

i cba

\end{document}
